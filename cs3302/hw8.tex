\documentclass[12pt]{article}
\usepackage{amsmath,amscd,amsbsy,amssymb,latexsym,url,bm,amsthm}
\usepackage{epsfig,graphicx,subfigure}
\usepackage[usenames]{xcolor}
\usepackage{hyperref}
\usepackage{listings}
%\hypersetup{hidelinks}
\usepackage[vlined,ruled,commentsnumbered,linesnumbered]{algorithm2e}
\providecommand{\abs}[1]{\lvert#1\rvert}
\providecommand{\norm}[1]{\lVert#1\rVert}

\newtheorem{thm}{Theorem}
\newtheorem{lemma}[thm]{Lemma}
\newtheorem{fact}[thm]{Fact}
\newtheorem{cor}[thm]{Corollary}
\newtheorem{eg}{Example}
\newtheorem{ex}{Exercise}
\newtheorem{defi}{Definition}
\newtheorem{hw}{Problem}
\newenvironment{sol}
  {\par\vspace{3mm}\noindent{\it Solution}.}
  {\qed}

\newcommand{\ov}{\overline}
\newcommand{\cb}{{\cal B}}
\newcommand{\cc}{{\cal C}}
\newcommand{\cd}{{\cal D}}
\newcommand{\ce}{{\cal E}}
\newcommand{\cf}{{\cal F}}
\newcommand{\ch}{{\cal H}}
\newcommand{\cl}{{\cal L}}
\newcommand{\cm}{{\cal M}}
\newcommand{\cp}{{\cal P}}
\newcommand{\cz}{{\cal Z}}
\newcommand{\eps}{\varepsilon}
\newcommand{\ra}{\rightarrow}
\newcommand{\la}{\leftarrow}
\newcommand{\Ra}{\Rightarrow}
\newcommand{\dist}{\mbox{\rm dist}}
\newcommand{\bn}{{\mathbf N}}
\newcommand{\bz}{{\mathbf Z}}

\setlength{\parindent}{0pt}
%\setlength{\parskip}{2ex}
\newenvironment{proofof}[1]{\bigskip\noindent{\itshape #1. }}{\hfill$\Box$\medskip}

\usepackage{enumerate,fullpage,proof}
\newcommand{\Infer}[2]{\infer{#2}{#1}}

\begin{document}

$\;$\hfill Due: 2022/4/9

\begin{center}
{\LARGE\bf Homework 8 - MM}
\end{center}

\begin{center}
	\footnotesize{\color{red}$*$ If there is any problem, please contact TA.}
	
	\footnotesize{\color{blue} \quad Name:\_\_\_\_\_\_\_\_\_  \quad Student ID:\_\_\_\_\_\_\_\_\_ \quad Email: \_\_\_\_\_\_\_\_\_\_\_\_}
\end{center}

\begin{hw}\rm (25 points)
	C has functions \texttt{malloc} and \texttt{free} that allow programmers to dynamically allocate and deallocate heap space. Research these two functions and compare their similarities and differences with the function \texttt{new} and \texttt{delete}.
\end{hw}

\begin{hw}\rm (25 points)
	Can you think of a better Mark-n-Sweep algorithm that reduces the amount of waiting time when garbage collecting?
\end{hw}

\begin{hw}\rm (50 points)
	Read the Java code program below and answer the question.
	
	\centering
	\begin{lstlisting}[language=Java] 
	Object[] p = new Object[2];
	Object[] q = new Object[2];
	Object[] s = new Object[2];
	Object[] t = new Object[2];
	p[0] = q;
	p[1] = s;
	q[0] = p;
	q[1] = t;
	s[0] = t;
	s[1] = null;
	t[0] = s;
	t[1] = null;
	for (int i = 0; i < 3; i++) {
	s = new Object[2];
	t = new Object[2];
	s[0] = t;
	s[1] = p;
	t[0] = s;
	t[1] = q;
	p[1] = null;
	q[1] = null;
	p = s;
	q = t;
	}
	\end{lstlisting} 
	
	\begin{enumerate}[(a)]
		\item Show the configuration of memory cells under reference counting.(After the final step)
		\item Suppose we use mark-and-sweep GC on the same program, and the maximum heap
		size is 20 memory cells (one cell for one object). When GC action happens, and after GC, what is the
		configuration of the memory?
	\end{enumerate}
\end{hw}



\vspace{20pt}

\textbf{Remark:} 
You need to use \textbf{LaTeX} to write your homework and \textbf{convert it into .pdf} file.

Please upload both \textbf{.tex and .pdf} files on \textbf{Canvas}.

File name format: {\color{red} HW\_X\_Name\_StudentID.tex/\color{red} HW\_X\_Name\_StudentID.pdf}

\end{document}
