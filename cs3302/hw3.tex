\documentclass[12pt]{article}
\usepackage{amsmath,amscd,amsbsy,amssymb,latexsym,url,bm,amsthm}
\usepackage{epsfig,graphicx,subfigure}
\usepackage[usenames]{xcolor}
\usepackage{hyperref}
%\hypersetup{hidelinks}
\usepackage[vlined,ruled,commentsnumbered,linesnumbered]{algorithm2e}
\providecommand{\abs}[1]{\lvert#1\rvert}
\providecommand{\norm}[1]{\lVert#1\rVert}

\newtheorem{thm}{Theorem}
\newtheorem{lemma}[thm]{Lemma}
\newtheorem{fact}[thm]{Fact}
\newtheorem{cor}[thm]{Corollary}
\newtheorem{eg}{Example}
\newtheorem{ex}{Exercise}
\newtheorem{defi}{Definition}
\newtheorem{hw}{Problem}
\newenvironment{sol}
  {\par\vspace{3mm}\noindent{\it Solution}.}
  {\qed}

\newcommand{\ov}{\overline}
\newcommand{\cb}{{\cal B}}
\newcommand{\cc}{{\cal C}}
\newcommand{\cd}{{\cal D}}
\newcommand{\ce}{{\cal E}}
\newcommand{\cf}{{\cal F}}
\newcommand{\ch}{{\cal H}}
\newcommand{\cl}{{\cal L}}
\newcommand{\cm}{{\cal M}}
\newcommand{\cp}{{\cal P}}
\newcommand{\cz}{{\cal Z}}
\newcommand{\eps}{\varepsilon}
\newcommand{\ra}{\rightarrow}
\newcommand{\la}{\leftarrow}
\newcommand{\Ra}{\Rightarrow}
\newcommand{\dist}{\mbox{\rm dist}}
\newcommand{\bn}{{\mathbf N}}
\newcommand{\bz}{{\mathbf Z}}

\setlength{\parindent}{0pt}
%\setlength{\parskip}{2ex}
\newenvironment{proofof}[1]{\bigskip\noindent{\itshape #1. }}{\hfill$\Box$\medskip}

\usepackage{enumerate,fullpage,proof}
\newcommand{\Infer}[2]{\infer{#2}{#1}}

\begin{document}

$\;$\hfill Due: 2022/3/5

\begin{center}
{\LARGE\bf Homework 3 - Lambda}
\end{center}

\begin{center}
	\footnotesize{\color{red}$*$ If there is any problem, please contact TA:yvonne\_huang@sjtu.edu.cn.}
	
	\footnotesize{\color{blue} \quad Name:\_\_\_\_\_\_\_\_\_  \quad Student ID:\_\_\_\_\_\_\_\_\_ \quad Email: \_\_\_\_\_\_\_\_\_\_\_\_}
\end{center}

\begin{hw}\rm (40 points)
	Evaluate the following $\lambda$ expressions using call-by-value and call-by-name. Show the complete steps of evaluation.
	\begin{enumerate}[(a)]
		\item $(\lambda x.\ ((\lambda y.\ x+z+3 )\ 3 )5)$
		\item $((\lambda v.(\lambda w.w))\ ((\lambda x.x)\ (y\ (\lambda z.z))))$
		\item $((\lambda x.\ x\ x)\ (\lambda y.\ y\ y))$
		\item $((\lambda x. \lambda y. x)\ (\lambda z.z\ \lambda u.u))$
	\end{enumerate}
\end{hw}

%\begin{sol}
%	content...
%\end{sol}


\begin{hw}\rm (30 points)
		
		Prove by induction: If $e_{1}$ is closed and $e_{1} \rightarrow e_{2}$, then $e_{2}$ is closed.
	Suppose using call by value evaluation. 
	\begin{itemize}
	\item Given the following definitions:
	\begin{enumerate}
		\item Rules of free variables (You can use these rules directly without writing "By ...")
		\[
		\frac{}{FV(x)=\{x\}} \qquad
		\frac{FV(e_1) = S_1\quad FV(e_2) = S_2}{FV(e_1\ e_2)=S_1\cup S_2} \qquad
		\frac{FV(e) = S}{FV(\\x.e)=S-\{x\}}
		\]
		\item Judgment form: \textbf{define} $e_1\rightarrow e_2$
		\[
		\frac{}{(\lambda x.e)\ v \rightarrow e [v/x]}
		\qquad
		\frac{e_1 \rightarrow e_1^{'}}{e_1\ e_2 \rightarrow e_1^{'}\ e_2}
		\qquad
		\frac{e_2 \rightarrow e_2^{'}}{v\ e_2 \rightarrow v\ e_2^{'}}
		\]
		
		\item Judgment form: \textbf{define} $e_1{\rightarrow}^*e_2$
		\[
		\frac{}{e_1\rightarrow^* e_1}
		\qquad
		\frac{e_1\rightarrow e_2\qquad e_2\rightarrow^*e_3}{e_1\rightarrow^*e_3}
		\]
	\end{enumerate}
	\item And given this lemma:
		\begin{lemma}\label{fv}
			$FV(e_1[e_2/x]) \subseteq (FV(e_1) - \{x\}) \cup FV(e_2)$
		\end{lemma}
	You can use this lemma directly. (Proof is in the appendix)
	\end{itemize}	
	
\end{hw}

%\begin{sol}
%	content...
%\end{sol}

\begin{hw}\rm (30 points)
	Church encoding is a means of embedding data and operators into the $\lambda$ calculus, the most familiar form being the Church numerals, a representation of the natural numbers using $\lambda$ notation. 
	Church numerals $\mathbf{0}, \mathbf{1}, \mathbf{2}, ...,$ are defined as follows:
	\begin{align*}
		\mathbf{0} &= \lambda f.\lambda x.\ x \\
		\mathbf{1} &= \lambda f.\lambda x.\ f\ x \\
		\mathbf{2} &= \lambda f.\lambda x.\ f\ (f\ x) \\
		\mathbf{3} &= \lambda f.\lambda x.\ f\ (f\ (f\ x)) \\
		&\dots \\
		\mathbf{n} &= \lambda f.\lambda x.\ f^n\ x \\
		&\dots
	\end{align*}
	Church numerals takes two parameters $f$ and $x$. Church numerals $n$ means apply $f$ to $x$ $n$ times.
	(You can refer to Wikipedia or other references about Church encoding to know more about the idea of church encoding)
	\begin{enumerate}[(a)]
		\item Define addition in $\lambda$ calculus, and then show the evaluation of $3+2$.
		\item Define multiplication in $\lambda$ calculus (Hint: use definition of addition), and then show the evaluation of $3\times2$.
		\item Give a definition of multiplication on Church numerals without using addition.
	\end{enumerate}
	
\end{hw}

\vspace{20pt}

\textbf{Remark:} 
You need to use \textbf{LaTeX} to write your homework and \textbf{convert it into .pdf} file.

Please upload both \textbf{.tex and .pdf} files on \textbf{Canvas}.

File name format: {\color{red} HW\_X\_Name\_StudentID.tex/\color{red} HW\_X\_Name\_StudentID.pdf}

\newpage

\appendix
\renewcommand{\appendixname}{Appendix~\Alph{section}}

\section{Proof of Lemma 1}


\textbf{Lemma}	$FV(e_1[e/x]) \subseteq (FV(e_1) - \{x\}) \cup FV(e)$


\begin{proof}
	By induction on the derivation of substitution. (Actually you can treat substitution as a special judgement form and the definitions as rules)
	\begin{enumerate}
		\item case $x[e/x] = e$
		
		Need to prove: $FV(x[e/x]) \subseteq (FV(x)-\{x\}) \cup FV(e)$
		
		(1) $FV(x[e/x])=FV(e)$ (by assumption)
		
		(2) $FV(e) \subseteq (FV(x) - \{x\}) \cup FV(e)$
		
		(3) $FV(x[e/x]) \subseteq (FV(x)-\{x\}) \cup FV(e)$ (by (1) and (2))
		
		\item case $y[e/x] = y$
		
		Need to prove: $FV(y[e/x]) \subseteq (FV(y)-\{x\}) \cup FV(e)$
		
		(1) $FV(y[e/x])=FV(y)$ (by assumption)
		
		(2) $FV(y) \subseteq FV(y)-\{x\} \subseteq (FV(y)-\{x\}) \cup FV(e)$
		
		(3) $FV(y[e/x]) \subseteq (FV(y)-\{x\}) \cup FV(e)$ (by (1) and (2))
		
		\item case $(e_1\ e_2)[e/x] = e_1[e/x]\ e_2[e/x]$ (rule format: $\frac{e_1[e/x]=a\ e_2[e/x]=b}{(e_1\ e_2)[e/x] = a\ b}$)
		
		Need to prove: $FV((e_1\ e_2)[e/x]) \subseteq (FV(e_1\ e_2) -\{x\}) \cup FV(e)$
		
		(1) $FV((e_1\ e_2)[e/x]) = FV(e_1[e/x]) \cup FV(e_2[e/x])$ (by assumption)
		
		(2) $FV(e_1[e/x]) \subseteq (FV(e_1) -\{x\}) \cup FV(e)$ (by I.H.)
		
		(3) $FV(e_2[e/x]) \subseteq (FV(e_2) -\{x\}) \cup FV(e)$ (by I.H.)
		
		(4) $FV((e_1\ e_2)[e/x]) \subseteq (FV(e_1) -\{x\}) \cup (FV(e_2) -\{x\}) \cup FV(e) \subseteq (FV(e_1\ e_2) -\{x\}) \cup FV(e)$ (by (1), (2) and (3))
		
		\item case $(\lambda x.e_1)[e/x] = \lambda x.e_1$
		
		Need to prove $FV((\lambda x.e_1)[e/x]) \subseteq (FV(\lambda x.e_1) -\{x\}) \cup FV(e)$
		
		(1) $FV((\lambda x.e_1)[e/x]) = FV(\lambda x.e_1)$ (by assumption)
		
		(2) $FV((\lambda x.e_1)[e/x]) \subseteq (FV(\lambda x.e_1) -\{x\}) \cup FV(e)$ (by (1))
		
		\item case $(\lambda y.e_1)[e/x] = \lambda y.(e_1[e/x])$ ($y \neq x$ and $y \notin FV(e)$)
		
		Need to prove $FV((\lambda y.e_1)[e/x]) \subseteq (FV(\lambda y.e_1) -\{x\}) \cup FV(e)$
		
		(1) $FV((\lambda y.e_1)[e/x]) = FV(\lambda y.(e_1[e/x])) = FV(e_1[e/x]) - \{y\}$ (by assumption)
		
		(2) $FV(e_1[e/x]) \subseteq (FV(e_1) - \{x\}) \cup FV(e)$ (by I.H.)
		
		(3) $FV(\lambda y.e_1) = FV(e_1) -\{y\}$
		
		(4) $FV((\lambda y.e_1)[e/x]) \subseteq (FV(\lambda y.e_1) -\{x\}) \cup FV(e)$ (by (1), (2) and (3))
		
		\item case $(\lambda y.e_1)[e/x] = \lambda z.(e_1[[z/y]][e/x])$ ($y \neq x$, $y \in FV(e)$ and $z \notin FV(e) \cup Vars(e_1)$)
		
		Need to prove $FV((\lambda y.e_1)[e/x]) \subseteq (FV(\lambda y.e_1) -\{x\}) \cup FV(e)$
		
		(1) $FV((\lambda y.e_1)[e/x]) = FV(\lambda z.(e_1[[z/y]][e/x])) = FV(e_1[[z/y]][e/x]) - \{z\}$ (by assumption)
		
		(2) $FV(e_1[[z/y]][e/x]) \subseteq (FV(e_1[[z/y]]) - \{x\}) \cup FV(e)$ (by I.H.)
		
		(3) $FV(e_1[[z/y]]) \subseteq (FV(e_1) - \{y\}) \cup FV(z)$ (by I.H.)
		
		(4) $FV(\lambda y.e_1) = FV(e_1) - \{y\}$
		
		(5) $FV((\lambda y.e_1)[e/x]) \subseteq (FV(\lambda y.e_1) -\{x\}) \cup FV(e)$ (by (1), (2), (3) and (4))
		
	\end{enumerate}
\end{proof}
About the proof of this Lemma, you just need to know the idea. I found another proof ( \href{https://www.cs.bham.ac.uk/research/projects/poplog/paradigms_lectures/lambda_calculus_html.d/substitution.html}{This is another proof}) from the internet. You will find the idea is the same. Both proofs use induction to prove this property.

\end{document}
