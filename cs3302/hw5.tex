\documentclass[12pt]{article}
\usepackage{amsmath,amscd,amsbsy,amssymb,latexsym,url,bm,amsthm}
\usepackage{epsfig,graphicx,subfigure}
\usepackage[usenames]{xcolor}
\usepackage{hyperref}
%\hypersetup{hidelinks}
\usepackage[vlined,ruled,commentsnumbered,linesnumbered]{algorithm2e}
\providecommand{\abs}[1]{\lvert#1\rvert}
\providecommand{\norm}[1]{\lVert#1\rVert}

\newtheorem{thm}{Theorem}
\newtheorem{lemma}[thm]{Lemma}
\newtheorem{fact}[thm]{Fact}
\newtheorem{cor}[thm]{Corollary}
\newtheorem{eg}{Example}
\newtheorem{ex}{Exercise}
\newtheorem{defi}{Definition}
\newtheorem{hw}{Problem}
\newenvironment{sol}
  {\par\vspace{3mm}\noindent{\it Solution}.}
  {\qed}

\newcommand{\ov}{\overline}
\newcommand{\cb}{{\cal B}}
\newcommand{\cc}{{\cal C}}
\newcommand{\cd}{{\cal D}}
\newcommand{\ce}{{\cal E}}
\newcommand{\cf}{{\cal F}}
\newcommand{\ch}{{\cal H}}
\newcommand{\cl}{{\cal L}}
\newcommand{\cm}{{\cal M}}
\newcommand{\cp}{{\cal P}}
\newcommand{\cz}{{\cal Z}}
\newcommand{\eps}{\varepsilon}
\newcommand{\ra}{\rightarrow}
\newcommand{\la}{\leftarrow}
\newcommand{\Ra}{\Rightarrow}
\newcommand{\dist}{\mbox{\rm dist}}
\newcommand{\bn}{{\mathbf N}}
\newcommand{\bz}{{\mathbf Z}}

\setlength{\parindent}{0pt}
%\setlength{\parskip}{2ex}
\newenvironment{proofof}[1]{\bigskip\noindent{\itshape #1. }}{\hfill$\Box$\medskip}

\usepackage{enumerate,fullpage,proof}
\newcommand{\Infer}[2]{\infer{#2}{#1}}

\begin{document}

$\;$\hfill Due: 2022/3/19

\begin{center}
{\LARGE\bf Homework 5 - Extend}
\end{center}

\begin{center}
	\footnotesize{\color{red}$*$ If there is any problem, please contact TA.}
	
	\footnotesize{\color{blue} \quad Name:\_\_\_\_\_\_\_\_\_  \quad Student ID:\_\_\_\_\_\_\_\_\_ \quad Email: \_\_\_\_\_\_\_\_\_\_\_\_}
\end{center}

\begin{hw}\rm (30 points)
	Consider the following program which is written in C syntax.
	\begin{verbatim}
	int x = 1;
	
	void f1() {
	int x = 2;
	f2();
	printf(x)
	}
	
	void f2() {
	int x = 3;
	printf(x)
	}
	
	int main() {
	f1();
	printf(x)
	}
	\end{verbatim}
	\begin{enumerate}[(a)]
		\item What will be printed after running \texttt{main()} when it uses static scoping?
		dynamic scoping?
	\end{enumerate}
\end{hw}

\begin{hw}\rm (40 points)
	Extend tuples to records, and write the (a) syntax and (b) semantic rules for records.
	Example usage:
	\begin{itemize}
		\item Elements are indexed by labels:
		\begin{itemize}
			\item $\{y=10\}$
			\item $\{id=1,salary=50000,active=\mathbf{true}\}$
		\end{itemize}
		\item The order of the record fields is insignificant:
		\begin{itemize}
			\item $\{y=10,x=5\}$ is the same as $\{x=5,y=10\}$
		\end{itemize}
		\item To access fields of a record:
		\begin{itemize}
			\item $a.id$
			\item $b.salary$
		\end{itemize}
	\end{itemize}
\end{hw}

\begin{hw}\rm (30 points)
	Another way of defining \textbf{let} as derived form might be to desugar it by "executing" it immediately-i.e., to regard \textbf{Let x=$t_1$ in $t_2$} as an abbreviation for the substituted body \textbf{$t_2[t_1/x]$}. Is this a good idea ?
\end{hw}

\vspace{20pt}

\textbf{Remark:} 
You need to use \textbf{LaTeX} to write your homework and \textbf{convert it into .pdf} file.

Please upload both \textbf{.tex and .pdf} files on \textbf{Canvas}.

File name format: {\color{red} HW\_X\_Name\_StudentID.tex/\color{red} HW\_X\_Name\_StudentID.pdf}

\end{document}
