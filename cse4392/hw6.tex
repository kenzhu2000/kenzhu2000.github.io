%----PrefaceImport----%
\documentclass{article}
\usepackage{fancyhdr}
\usepackage[a4paper,margin=1in,headsep=25pt]{geometry}
\usepackage{lipsum,hyperref}
\usepackage{enumerate,fullpage,proof}
\usepackage[fontsize=12pt]{fontsize}
\usepackage{amsmath,amscd,amsbsy,amssymb,latexsym,url,bm,amsthm}
\usepackage{epsfig,graphicx,subfigure}
\usepackage{listings}
\usepackage[usenames]{xcolor}
\usepackage{tcolorbox}

\newtheorem{thm}{Theorem}
\newtheorem{lemma}[thm]{Lemma}

\pagestyle{fancy}
\pagenumbering{Alph}
\setlength{\headheight}{36.0pt}
\headsep = 25pt
\fancyhf{}
\lhead{CSE 4392 Special Topic: Natural Language Processing}
\rhead{Homework 5: Word Embeddings}
\lfoot{2025 Kenny Zhu & Essam Abdelghany}
\rfoot{Plagiarism will not be tolerated. We are here to help.}


% New command for blank spaces after questions
\newcommand{\answerbox}{
    \vspace{7cm} % Adjust the space size as needed
}
\newcommand{\answerboxbig}{
    \vspace{20cm} % Adjust the space size as needed
}
\newcommand{\answerboxsmall}{
    \vspace{3cm} % Adjust the space size as needed
}


%----Documentation----%
\begin{document}

\title{CSE 4392 Special Topic: Natural Language Processing}
\author{Homework 6 - Spring 2025}
\date{Due Date: Mar 3, 2025, 11:59 p.m. Central Time}
\maketitle
\thispagestyle{fancy}

%----Homeworks----%
\section*{Problem 1 (40\%)}

Consider a neural network with the following hypothesis function:
\begin{align*}
\textbf{Input:} \quad & \mathbf{x}, \\
\mathbf{h}_1 &= \tanh\Bigl(\mathbf{W}_1 \mathbf{x} + \mathbf{b}_1\Bigr), \\
\mathbf{h}_2 &= \tanh\Bigl(\mathbf{W}_2 \mathbf{h}_1 + \mathbf{b}_2\Bigr), \\
\mathbf{y} &= \sigma\Bigl(\mathbf{w}^\top \mathbf{h}_2 + b\Bigr),
\end{align*}
and the loss function is given by
\[
\mathcal{L}(\mathbf{y}, y^*) = -y^*\log y - (1-y^*)\log (1-y).
\]
\\[10pt]
\textbf{Task:} Compute the gradient of $\mathcal{L}$ with respect to each parameter in the network, namely:
\[
\mathbf{W}_1,\quad \mathbf{b}_1,\quad \mathbf{W}_2,\quad \mathbf{b}_2,\quad \mathbf{w},\quad \text{and} \quad b.
\]
Your solution must include four sections:
\begin{enumerate}
    \item \textbf{Chain Rule Expression:} Write the full chain rule expression for each gradient.
    \item \textbf{Differentiation Identities:} List the differentiation identities expected to be used to compute the chain terms.
    \item \textbf{Step-by-Step Computation:} Compute each chain term step by step, referencing the identity used.
    \item \textbf{Final Gradient Equations:} Present the final gradient and thereby show the gradient descent update equation.
\end{enumerate}
\\[10pt]
Ensure an organized solution to prevent losing marks. It's good to start with a draft and then rewrite in an organized way or just ust Latex!
\section*{Problem 2 - 60\%}
In this problem, you will leverage your understanding of neural network fundamentals to build a simple fully connected neural network for classifying MBTI (Myers-Briggs Type Indicator) personality types based on preprocessed posts.
\\
\newline
\textbf{dataset}: \href{https://drive.google.com/file/d/1CHuGdZKJJRl_IRdxa3MBv4SqEJVUhK73/view?usp=sharing}{MBTI 500.csv}
\\
\textbf{columns}:
\begin{itemize}
    \item \textbf{posts}: Equal-sized posts with 500 words per sample.
    \item \textbf{type}: MBTI personality types indicating differing psychological preferences.
\end{itemize}
\textbf{instructions}:
\begin{enumerate}
    \item \textbf{Dataset Exploration:}
        \begin{itemize}
            \item Load the provided CSV file (\textbf{MBTI500.csv}). Split it into training and testing splits after randomizing it. It's sufficient for the test set to include 5K examples.
        \end{itemize}
    \item \textbf{Data Preprocessing:}
        \begin{itemize}
            \item Decide whether to perform a 4-way classification (based on dichotomies) or a 16-way classification (considering each personality type independently). Your decision will guide how you preprocess the type column.
        \end{itemize}
    \item \textbf{Feature Extraction:}
        \begin{itemize}
            \item Utilize existing libraries to extract features from the posts to map each of them into a vector.
        \end{itemize}
    \item \textbf{Neural Network Construction:}
        \begin{itemize}
            \item Use PyTorch to build a neural network for this task 
        \end{itemize}
    \item \textbf{Training:}
        \begin{itemize}
            \item Tune different hyperparameters of the neural network to optimize it for this task. 
        \end{itemize}
    \item \textbf{Evaluation:}
        \begin{itemize}
            \item For each hyperparameter setting, evaluate the model using Macro F1, Macro Precision, Macro Recall and Accuracy. Show results in the report for at least 7 different hyperparameter configurations.
        \end{itemize}
    \item \textbf{Documentation:}
        \begin{itemize}
            \item Create a PDF report detailing your approach, including decisions made during data preprocessing, feature extraction, and network architecture. Provide insights into your model's performance and discuss any challenges faced.
        \end{itemize}
    \item \textbf{Attach your codes and report.}
\end{enumerate}
Evaluation will partly be part of the best evaluation result of your model. Do not zip your files before submission.

\end{document}
