%----PrefaceImport----%
\documentclass{article}
\usepackage{fancyhdr}
\usepackage[a4paper,margin=1in,headsep=25pt]{geometry}
\usepackage{lipsum,hyperref}
\usepackage{enumerate,fullpage,proof}
\usepackage[fontsize=12pt]{fontsize}
\usepackage{amsmath,amscd,amsbsy,amssymb,latexsym,url,bm,amsthm}
\usepackage{epsfig,graphicx,subfigure}
\usepackage{listings}
\usepackage[usenames]{xcolor}
\usepackage{tcolorbox}

\newtheorem{thm}{Theorem}
\newtheorem{lemma}[thm]{Lemma}

\pagestyle{fancy}
\pagenumbering{Alph}
\setlength{\headheight}{36.0pt}
\headsep = 25pt
\fancyhf{}
\lhead{CSE 4392 Special Topic: Natural Language Processing}
\rhead{Homework 4: Logistic Regression}
\lfoot{2025 Kenny Zhu & Essam Abdelghany}
\rfoot{Plagiarism will not be tolerated. We are here to help.}


% New command for blank spaces after questions
\newcommand{\answerbox}{
    \vspace{7cm} % Adjust the space size as needed
}
\newcommand{\answerboxbig}{
    \vspace{20cm} % Adjust the space size as needed
}
\newcommand{\answerboxsmall}{
    \vspace{3cm} % Adjust the space size as needed
}


%----Documentation----%
\begin{document}

\title{CSE 4392 Special Topic: Natural Language Processing}
\author{Homework 4 - Spring 2025}
\date{Due Date: Feb 19, 2025, 11:59 p.m. Central Time}
\maketitle
\thispagestyle{fancy}

%----Homeworks----%

\section*{Problem 1 - 60\%}
Finish the todos for implementing logistic regression from scratch in \textbf{Logistic.ipynb}. The accuracy over the toy dataset should be 99\% if your implementation is correct. After finishing upload your filled \textbf{Logistic.ipynb} and an equivalent \textbf{Logistic.pdf}  which can be produced using a tool such as \href{https://htmtopdf.herokuapp.com/ipynbviewer/}{this}.



\section*{Problem 2 - 40\%}
Your task is to build a document classification model to classify product descriptions from an E-commerce website into four categories: Electronics, Household, Books, and Clothing \& Accessories.  
\\[10pt]  
You will use the provided dataset in CSV format, which contains two columns: the first column represents the class name, and the second column represents the corresponding product description.  
You can find the dataset here: \href{https://drive.google.com/file/d/1YfG0iy0vpv7L0aqzDxvzHlW79HDdpZvg/view?usp=drive_link}{Dataset Link}
\\[10pt]  
You will use the logistic model you implement from scratch (make a copy of the file called \textbf{Ecommerce.ipynb} and do away with the toy example). Your notebook should have five main sections aside from the logistic implementation: dataset loading and splitting, feature extraction, feature visualization (optional), model training, model evaluation (using Macro F1 score and accuracy).
\\[10pt]  
The most important section will be feature extraction where you will implement a procedure to map any text document from the dataset into a numerical vector and then apply it on all train and test examples.
\\[10pt]  
After finishing, upload your filled \textbf{Ecommerce.ipynb} and an equivalent \textbf{Ecommerce.pdf} which can be produced using a tool such as \href{https://htmtopdf.herokuapp.com/ipynbviewer/}{this}.

\\[10pt]  
\begin{tcolorbox}[colback=blue!5, colframe=blue!75!black, title=Note]
If your accuracy/F1 is not reasonable (or is exceptional!) then this may affect evaluation. Submitting on or before Feb. 18th grants you a 7\% bonus.
\end{tcolorbox}
\end{document}
