\documentclass[12pt]{article}
\usepackage{amsmath,amscd,amsbsy,amssymb,latexsym,url,bm,amsthm}
\usepackage{epsfig,graphicx,subfigure}
\usepackage[usenames]{xcolor}
\usepackage{hyperref}
%\hypersetup{hidelinks}
\usepackage[vlined,ruled,commentsnumbered,linesnumbered]{algorithm2e}
\providecommand{\abs}[1]{\lvert#1\rvert}
\providecommand{\norm}[1]{\lVert#1\rVert}

\newtheorem{thm}{Theorem}
\newtheorem{lemma}[thm]{Lemma}
\newtheorem{fact}[thm]{Fact}
\newtheorem{cor}[thm]{Corollary}
\newtheorem{eg}{Example}
\newtheorem{ex}{Exercise}
\newtheorem{defi}{Definition}
\newtheorem{hw}{Problem}
\newenvironment{sol}
  {\par\vspace{3mm}\noindent{\it Solution}.}
  {\qed}

\newcommand{\ov}{\overline}
\newcommand{\cb}{{\cal B}}
\newcommand{\cc}{{\cal C}}
\newcommand{\cd}{{\cal D}}
\newcommand{\ce}{{\cal E}}
\newcommand{\cf}{{\cal F}}
\newcommand{\ch}{{\cal H}}
\newcommand{\cl}{{\cal L}}
\newcommand{\cm}{{\cal M}}
\newcommand{\cp}{{\cal P}}
\newcommand{\cz}{{\cal Z}}
\newcommand{\eps}{\varepsilon}
\newcommand{\ra}{\rightarrow}
\newcommand{\la}{\leftarrow}
\newcommand{\Ra}{\Rightarrow}
\newcommand{\dist}{\mbox{\rm dist}}
\newcommand{\bn}{{\mathbf N}}
\newcommand{\bz}{{\mathbf Z}}

\setlength{\parindent}{0pt}
%\setlength{\parskip}{2ex}
\newenvironment{proofof}[1]{\bigskip\noindent{\itshape #1. }}{\hfill$\Box$\medskip}

\usepackage{enumerate,fullpage,proof}
\newcommand{\Infer}[2]{\infer{#2}{#1}}

\begin{document}

$\;$\hfill Due: 2022/4/2

\begin{center}
{\LARGE\bf Homework 7 - IMP}
\end{center}

\begin{center}
	\footnotesize{\color{red}$*$ If there is any problem, please contact TA.}
	
	\footnotesize{\color{blue} \quad Name:\_\_\_\_\_\_\_\_\_  \quad Student ID:\_\_\_\_\_\_\_\_\_ \quad Email: \_\_\_\_\_\_\_\_\_\_\_\_}
\end{center}

\begin{hw}\rm (30 points)

	Wouldn't it be simpler just to require the programmer to annotate error with its intended type in each context where it is used ? Why ?
\end{hw}


\begin{hw} \rm (35 points)

  In lecture \emph{Going Imperative}, the language is extended with while loop.
	In this problem, you are required to define the syntax and the semantics
	(including evaluation rules and typing rules) of while loop with \texttt{break} and \texttt{continue}
\end{hw}

\begin{hw} \rm (35 points)

	Proof \textbf{Preservation Theorem}: If $\Sigma;\Gamma \vdash e:t, \Sigma;\Gamma \vdash M$, and $(M,e)\rightarrow(M',e')$, then for some $\Sigma' \supseteq \Sigma, \Sigma';\Gamma \vdash e':t, \Sigma';\Gamma \vdash M'$. ($\Sigma' \supseteq \Sigma$ means $\Sigma'$ agrees with $\Sigma$ on all the old locations.)
	
	Hint: You don't need to write ``need to prove..." in this problem since in all cases it's quite similar. Also, you can use directly the following two lemma whose proofs are quite easy:
	\begin{lemma}
	
	\textbf{Substitution:} If $\Sigma;\Gamma, x:t_1 \vdash e:t_2$ and $\Sigma;\Gamma \vdash v:t_1$, then $\Sigma;\Gamma \vdash e[v/x]:t_2$
	(similar to the proof of previous substitution lemma)
\end{lemma}

\begin{lemma}
	If $\Sigma;\Gamma \vdash e:t$ and $\Sigma' \supseteq \Sigma$, then $\Sigma';\Gamma \vdash e:t$
	(easy induction)
\end{lemma}
	
\end{hw}


\vspace{20pt}

\textbf{Remark:} 
You need to use \textbf{LaTeX} to write your homework and \textbf{convert it into .pdf} file.

Please upload both \textbf{.tex and .pdf} files on \textbf{Canvas}.

File name format: {\color{red} HW\_X\_Name\_StudentID.tex/\color{red} HW\_X\_Name\_StudentID.pdf}
\end{document}
