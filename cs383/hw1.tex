\documentclass[12pt]{article}
\usepackage{amsmath,amscd,amsbsy,amssymb,latexsym,url,bm,amsthm}
\usepackage{epsfig,graphicx,subfigure}
\usepackage[usenames]{xcolor}
\usepackage{hyperref}
%\hypersetup{hidelinks}
\usepackage[vlined,ruled,commentsnumbered,linesnumbered]{algorithm2e}
\providecommand{\abs}[1]{\lvert#1\rvert}
\providecommand{\norm}[1]{\lVert#1\rVert}

\newtheorem{thm}{Theorem}
\newtheorem{lemma}[thm]{Lemma}
\newtheorem{fact}[thm]{Fact}
\newtheorem{cor}[thm]{Corollary}
\newtheorem{eg}{Example}
\newtheorem{ex}{Exercise}
\newtheorem{defi}{Definition}
\newtheorem{hw}{Problem}
\newenvironment{sol}
  {\par\vspace{3mm}\noindent{\it Solution}.}
  {\qed}

\newcommand{\ov}{\overline}
\newcommand{\cb}{{\cal B}}
\newcommand{\cc}{{\cal C}}
\newcommand{\cd}{{\cal D}}
\newcommand{\ce}{{\cal E}}
\newcommand{\cf}{{\cal F}}
\newcommand{\ch}{{\cal H}}
\newcommand{\cl}{{\cal L}}
\newcommand{\cm}{{\cal M}}
\newcommand{\cp}{{\cal P}}
\newcommand{\cz}{{\cal Z}}
\newcommand{\eps}{\varepsilon}
\newcommand{\ra}{\rightarrow}
\newcommand{\la}{\leftarrow}
\newcommand{\Ra}{\Rightarrow}
\newcommand{\dist}{\mbox{\rm dist}}
\newcommand{\bn}{{\mathbf N}}
\newcommand{\bz}{{\mathbf Z}}

\setlength{\parindent}{0pt}
%\setlength{\parskip}{2ex}
\newenvironment{proofof}[1]{\bigskip\noindent{\itshape #1. }}{\hfill$\Box$\medskip}

\usepackage{enumerate,fullpage,proof}
\newcommand{\Infer}[2]{\infer{#2}{#1}}

\begin{document}

$\;$\hfill Due: 2022/2/19 23:59

\begin{center}
{\LARGE\bf Homework 1 - Overview}
\end{center}

\begin{center}
	\footnotesize{\color{red}$*$ If there is any problem, please contact TA: yvonne\_huang@sjtu.edu.cn}

	\footnotesize{\color{blue} \quad Name:\_\_\_\_\_\_\_\_\_  \quad Student ID:\_\_\_\_\_\_\_\_\_ \quad Email: \_\_\_\_\_\_\_\_\_\_\_\_}
\end{center}

\begin{hw}\rm (30 points)
	Give a feature of C, C++ or Java that illustrates orthogonality. Give a feature that illustrates non-orthogonality.
\end{hw}

%\begin{sol}
%	content...
%\end{sol}

\begin{hw}\rm (30 points)
	Write a Java function called \textit{SpOdd}. The function takes an array of integers as input and return an array of integers with all odd numbers in original array. Keep the same order as the original array. Then test your function in main function.
	
Sample output of main function:

\fbox{%

  \parbox{1\textwidth}{
Original Array: [3, 8, 5, 7, 1, 9, 2]
                                                                       
Odd elements in the array: [3, 5, 7, 1, 9]                              

  }

}
Please submit your .java file.

\end{hw}

%\begin{sol}
%	content...
%\end{sol}

\begin{hw}\rm (40 points)
We have learned the difference between compilers and interpreters. 
Now research compiled languages and interpreted languages. Then list the advantages 
and disadvantages of these two types of languages.
\end{hw}

%\begin{sol}
%	content...
%\end{sol}

\vspace{20pt}

\textbf{Remark:} 
You need to use \textbf{LaTeX} to write your homework and \textbf{convert it into .pdf} file.

Please upload both \textbf{.tex and .pdf} files on \textbf{Canvas}. For problem 2, you also need to submit a \textbf{.java} file.

File name format: {\color{red} HW\_X\_Name\_StudentID.tex/\color{red} HW\_X\_Name\_StudentID.pdf}

\end{document}
