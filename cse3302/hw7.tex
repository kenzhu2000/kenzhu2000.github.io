%----PrefaceImport----%
\documentclass{article}
\usepackage{fancyhdr}
\usepackage[a4paper,margin=1in,headsep=25pt]{geometry}
\usepackage{lipsum,hyperref}
\usepackage{enumerate,fullpage,proof}
\usepackage[fontsize=12pt]{fontsize}
\usepackage{amsmath,amscd,amsbsy,amssymb,latexsym,url,bm,amsthm}
\usepackage{epsfig,graphicx,subfigure}
\usepackage[usenames]{xcolor}
\usepackage{tcolorbox}
\usepackage{amsmath}
\usepackage{listings}
\usepackage{xcolor}

\newtheorem{thm}{Theorem}
\newtheorem{lemma}[thm]{Lemma}

\pagestyle{fancy}
\pagenumbering{Alph}
\setlength{\headheight}{36.0pt}
\headsep = 25pt
\fancyhf{}
\lhead{CSE 3302/5307 Programming Language Concepts}
\rhead{Homework 7: Going Imperative}
\lfoot{2024 Kenny Zhu & Essam Abdelghany}
\rfoot{Plagiarism will not be tolerated. We are here to help.}

\lstset{
  language=[Objective]Caml,
  basicstyle=\ttfamily,
  keywordstyle=\color{blue},
  commentstyle=\color{gray},
  stringstyle=\color{green},
  showstringspaces=false,
  numbers=left,
  numberstyle=\tiny,
  breaklines=true,
  frame=single
}

% New command for blank spaces after questions
\newcommand{\answerbox}{
    \vspace{7cm} % Adjust the space size as needed
}
\newcommand{\answerboxbig}{
    \vspace{20cm} % Adjust the space size as needed
}
\newcommand{\answerboxsmall}{
    \vspace{3cm} % Adjust the space size as needed
}

% New command for personal info at the end of the document
\newcommand{\studentinfo}{
    \noindent Name: \underline{\hspace{5cm}} UTA ID: \underline{\hspace{5cm}}\\
    \vspace{0.5cm} % Space after the fields
}

\usepackage{listings}
\usepackage{xcolor}

\lstset{
    language=C,
    basicstyle=\ttfamily\footnotesize,
    keywordstyle=\color{blue},
    commentstyle=\color{gray},
    stringstyle=\color{orange},
    numbers=none, % Hides line numbers
    showstringspaces=false,
    breaklines=true,
    frame=single,
    rulecolor=\color{black},
}


%----Documentation----%
\begin{document}

\title{CSE 3302/5307 Programming Language Concepts}
\author{Homework7 - Fall 2023}
\date{Due Date: Oct. 14th, 2024, 11:59p.m. Central Time}
\maketitle
\thispagestyle{fancy}

%----Homeworks----%

\section*{Problem1 - 30\%}

Explain using operational and typing inference rules covered in the lecture that the following satisfies both progress and preservation after defining each of them $(\lambda x:bool . x) \: error$. Why is it not good to force errors to be of a new specific type (e.g., $t:=err$)?



\newpage 



\section*{Problem2 - 30\%}

Extend the $while$ loop covered in the lecture to a $do-while$ loop. Include both operational and typing semantics. Write a code sample using your syntax and show three evaluation steps.

\newpage 


\section*{Problem3 - 30\%}
Evaluate the following showing the memory state in each step:

\begin{lstlisting}
let x = ref 5 in
let y = ref (!x * 2) in
x := (!x) - 2;
y := (!y) + (!x);
!y
\end{lstlisting}




\section*{Problem4 - 10\%}
In terms of functional programming, what is a pure function? Why are pure functions less bug prone to easier to reason about? You can research online for this question but you are responsible if the answer is wrong (it suffices to read from different sources and/or reach out if in doubt).


\section*{}
\answerboxbig

\studentinfo


\end{document}