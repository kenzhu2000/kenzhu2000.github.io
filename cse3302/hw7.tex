%----PrefaceImport----%
\documentclass{article}
\usepackage{fancyhdr}
\usepackage[a4paper,margin=1in,headsep=25pt]{geometry}
\usepackage{lipsum,hyperref}
\usepackage{enumerate,fullpage,proof}
\usepackage[fontsize=12pt]{fontsize}
\usepackage{amsmath,amscd,amsbsy,amssymb,latexsym,url,bm,amsthm}
\usepackage{epsfig,graphicx,subfigure}
\usepackage[usenames]{xcolor}
\usepackage{tcolorbox}
\usepackage{amsmath}

\newtheorem{thm}{Theorem}
\newtheorem{lemma}[thm]{Lemma}

\pagestyle{fancy}
\pagenumbering{Alph}
\setlength{\headheight}{36.0pt}
\headsep = 25pt
\fancyhf{}
\lhead{CSE 3302/5307 Programming Language Concepts}
\rhead{Homework 7: Simply-typed Extensions II}
\lfoot{2025 Kenny Zhu Wonjun Park}
\rfoot{Plagiarism will not be tolerated. We are here to help.}

% New command for blank spaces after questions
\newcommand{\answerbox}{
    \vspace{7cm} % Adjust the space size as needed
}
\newcommand{\answerboxbig}{
    \vspace{20cm} % Adjust the space size as needed
}
\newcommand{\answerboxsmall}{
    \vspace{3cm} % Adjust the space size as needed
}

% New command for personal info at the end of the document
\newcommand{\studentinfo}{
    $$\begin{array}{cc}
        \noindent \text{Name:} \underline{\hspace{5cm}} &
            \text{UTA ID:} \underline{\hspace{5cm}}\\
    \end{array}$$
}

\usepackage{listings}
\usepackage{xcolor}

\lstset{
    language=C,
    basicstyle=\ttfamily\footnotesize,
    keywordstyle=\color{blue},
    commentstyle=\color{gray},
    stringstyle=\color{orange},
    numbers=none, % Hides line numbers
    showstringspaces=false,
    breaklines=true,
    frame=single,
    rulecolor=\color{black},
}


%----Documentation----%
\begin{document}

\title{CSE 3302/5307 Programming Language Concepts}
\author{Homework 7 - Fall 2025}
\date{Due Date: Oct. 6, 2025, 9:00PM Central Time}
\maketitle
\thispagestyle{fancy}

\studentinfo

%----Homeworks----%

\section*{Problem1 - 30\%}

	We've seen how to define natural numbers using church encoding in untyped lambda calculus:
	\begin{align*}
		\mathbf{0} &= \lambda f.\lambda x.\ x \\
		\mathbf{1} &= \lambda f.\lambda x.\ f\ x \\
		&\dots \\
		\mathbf{n} &= \lambda f.\lambda x.\ f^n\ x \\
		&\dots
	\end{align*}
	Note that church encoding cannot represent negative integers.
	
	\begin{enumerate}[(a)]
		\item Propose a method to extend church numerals to representation of integers. Give a concrete example for representation of integer \textbf{-5} with your proposed method. Hint: you may try to use pairs.
		
		\item Define a function $nat2int$ that converts a natural number to your representation of correspondent integer.
		
		\item Based on this definition of integers, define the following arithmetic operations in lambda calculus(you can directly use operations on natural numbers defined before like add, etc. ): 
		\begin{enumerate}[(1)]
			\item negation: neg n
			\item addition: addint m n
			\item subtraction: subint m n
		\end{enumerate}
	\end{enumerate}
\answerboxbig


	% Force a new page and guarantee vertical space even at the top of the page
\newpage
\section*{Problem 2 - 30\%}

Given the definition of Fibonacci number
\[F_0 = 0, F_1 = 1, F_i = F_{i-1} + F_{i-2}\]

\begin{enumerate}[(a)]
	\item Use \textit{fix} to write a lambda function called \textit{fib}: int $\rightarrow$ int to compute the n-th Fibonacci number.
	
	\item We want to extend simple \textit{let} expression to recursive \textit{let rec} expression:
	\[letrec\ f = \lambda x.\ e_{1}\ in\ e_{2}\]
	where f itself can appear in $e_{1}$. 
	
	Example usage of \textit{letrec} for factorial:
	\[fact = \lambda n. (letrec\ fact= (\lambda i.\ if\ i=0\ then\ 1\ else\ i*(fact\ (i-1))) in\ fact\ n)\]
	
	\begin{enumerate}[(1)]
		\item Define semantic and typing rules for expression \textit{letrec} ;
		\item Use \textit{letrec} to redefine our Fibonacci function.
	\end{enumerate}
\end{enumerate}

\answerboxbig

\newpage
\section*{Problem 3 - 40\%}

	Given the following $\lambda$ expression:
 
	\begin{verbatim}
	let x = 2 in
	  let y = 4 in
	    let f1 = \x.\y.x+2*y in
	      let f2 = \x.\y.2*x-y in
	      f2 (f1 y x) 3
	\end{verbatim}
	
	Using the environment model for lambda calculus with let,
	
	\begin{enumerate}[(a)]
		\item Define closures. (Be careful and refer to lecture slides);
		 
		\item Show detailed multi-step evaluation process of the $\lambda$ expression above.
	\end{enumerate}
 
	(The environment should be clearly shown in each step)

\answerboxbig

\end{document}
