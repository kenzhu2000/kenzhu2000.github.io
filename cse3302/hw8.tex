%----PrefaceImport----%
\documentclass{article}
\usepackage{fancyhdr}
\usepackage[a4paper,margin=1in,headsep=25pt]{geometry}
\usepackage{lipsum,hyperref}
\usepackage{enumerate,fullpage,proof}
\usepackage[fontsize=12pt]{fontsize}
\usepackage{amsmath,amscd,amsbsy,amssymb,latexsym,url,bm,amsthm}
\usepackage{epsfig,graphicx,subfigure}
\usepackage[usenames]{xcolor}
\usepackage{tcolorbox}
\usepackage{amsmath}
\usepackage{listings}
\usepackage{xcolor}
\usepackage{enumitem}   % For customizing the enumerate environment

\newtheorem{thm}{Theorem}
\newtheorem{lemma}[thm]{Lemma}

\pagestyle{fancy}
\pagenumbering{Alph}
\setlength{\headheight}{36.0pt}
\headsep = 25pt
\fancyhf{}
\lhead{CSE 3302/5307 Programming Language Concepts}
\rhead{Homework 8: Memory Management}
\lfoot{2024 Kenny Zhu & Essam Abdelghany}
\rfoot{Plagiarism will not be tolerated. We are here to help.}

\lstset{
  language=[Objective]Caml,
  basicstyle=\ttfamily,
  keywordstyle=\color{blue},
  commentstyle=\color{gray},
  stringstyle=\color{green},
  showstringspaces=false,
  numbers=left,
  numberstyle=\tiny,
  breaklines=true,
  frame=single
}

% New command for blank spaces after questions
\newcommand{\answerbox}{
    \vspace{7cm} % Adjust the space size as needed
}
\newcommand{\answerboxbig}{
    \vspace{20cm} % Adjust the space size as needed
}
\newcommand{\answerboxsmall}{
    \vspace{3cm} % Adjust the space size as needed
}

% New command for personal info at the end of the document
\newcommand{\studentinfo}{
    \noindent Name: \underline{\hspace{5cm}} UTA ID: \underline{\hspace{5cm}}\\
    \vspace{0.5cm} % Space after the fields
}

\usepackage{listings}
\usepackage{xcolor}
\usepackage{placeins}

\lstset{
    language=C,
    basicstyle=\ttfamily\footnotesize,
    keywordstyle=\color{blue},
    commentstyle=\color{gray},
    stringstyle=\color{orange},
    numbers=none, % Hides line numbers
    showstringspaces=false,
    breaklines=true,
    frame=single,
    rulecolor=\color{black},
}


%----Documentation----%
\begin{document}

\title{CSE 3302/5307 Programming Language Concepts}
\author{Homework8 - Fall 2023}
\date{Due Date: Oct. 21st, 2024, 11:59p.m. Central Time}
\maketitle
\thispagestyle{fancy}

%----Homeworks----%

\section*{Problem1 - 60\%}

\FloatBarrier % This ensures no floating objects (figures) appear before this point
\begin{figure}[h]
    \centering
    \includegraphics[width=0.95\textwidth]{Example.png}
    \caption{Heap Configuration}
    \label{fig:example}
\end{figure}

\noindent
Consider the heap configuration shown for allocated memory words. The heap is in total 100B in this case and one memory word is 4B.
\begin{enumerate}[label=\alph*)]
    \item Compute the amount of memory leaked for this configuration
    \item Mark the nodes that will be added to the free\_list after reference counting
    \item Compute the amount of memory leaked after each of reference counting and mark and sweep
    \item How does mark and sweep detect the cycle although it is not in the reference graph?
    \item Compute the memory overhead from each of reference counting and mark and sweep. Assume that each count needed by RC takes 1 byte.
    \item Mention a set of conditions under which reference counting could be viewed to be significantly better than mark and sweep
\end{enumerate}

\answerboxbig

\newpage 
\mbox{}
\newpage 


\section*{Problem2 - 10\%}

In Cheney's algorithm, the memory words in the from\_space are mapped in a consistent and contiguous sense to the to\_space. If that's the case, why is it necessary to store the forward addresses while mapping? Consider illustrating your explanation with a diagram.



\newpage 


\section*{Problem3 - 30\%}
Prove the preservation theorem as covered in the lecture (Untyped Lambda Calculus II). For each case where the proof proceeds, mention the need to prove and inductive hypotheses in if-then statements. 


\section*{}
\answerboxbig

\studentinfo


\end{document}