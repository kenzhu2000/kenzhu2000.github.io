%----PrefaceImport----%
\documentclass{article}
\usepackage{fancyhdr}
\usepackage[a4paper,margin=1in,headsep=25pt]{geometry}
\usepackage{lipsum,hyperref}
\usepackage{enumerate,fullpage,proof}
\usepackage[fontsize=12pt]{fontsize}
\usepackage{amsmath,amscd,amsbsy,amssymb,latexsym,url,bm,amsthm}
\usepackage{epsfig,graphicx,subfigure}
\usepackage[usenames]{xcolor}
\usepackage{tcolorbox}
\usepackage{amsmath}
\usepackage{listings}
\usepackage{xcolor}

\newtheorem{thm}{Theorem}
\newtheorem{lemma}[thm]{Lemma}

\pagestyle{fancy}
\pagenumbering{Alph}
\setlength{\headheight}{36.0pt}
\headsep = 25pt
\fancyhf{}
\lhead{CSE 3302/5307 Programming Language Concepts}
\rhead{Homework 8: Going Imperative}
\lfoot{2025 Kenny Zhu Wonjun Park}
\rfoot{Plagiarism will not be tolerated. We are here to help.}

\lstset{
  language=[Objective]Caml,
  basicstyle=\ttfamily,
  keywordstyle=\color{blue},
  commentstyle=\color{gray},
  stringstyle=\color{green},
  showstringspaces=false,
  numbers=left,
  numberstyle=\tiny,
  breaklines=true,
  frame=single
}

% New command for blank spaces after questions
\newcommand{\answerbox}{
    \vspace{7cm} % Adjust the space size as needed
}
\newcommand{\answerboxbig}{
    \vspace{20cm} % Adjust the space size as needed
}
\newcommand{\answerboxsmall}{
    \vspace{3cm} % Adjust the space size as needed
}

\newcommand{\studentinfo}{
    $$\begin{array}{cc}
        \noindent \text{Name:} \underline{\hspace{5cm}} &
            \text{UTA ID:} \underline{\hspace{5cm}}\\
    \end{array}$$
}

\usepackage{listings}
\usepackage{xcolor}

\lstset{
    language=C,
    basicstyle=\ttfamily\footnotesize,
    keywordstyle=\color{blue},
    commentstyle=\color{gray},
    stringstyle=\color{orange},
    numbers=none, % Hides line numbers
    showstringspaces=false,
    breaklines=true,
    frame=single,
    rulecolor=\color{black},
}


%----Documentation----%
\begin{document}

\title{CSE 3302/5307 Programming Language Concepts}
\author{Homework 8 - Fall 2025}
\date{Due Date: Oct. 13, 2025, 9:00PM Central Time}
\maketitle
\thispagestyle{fancy}

\studentinfo

%----Homeworks----%
% Don't include the proof on progress and preservation (change it into another question)
\section*{Problem1 - 30\%}
Evaluate the following showing the memory state in each step:

\begin{lstlisting}
let x = ref 5 in
    let y = ref (!x * 2) in
        x := (!x) - 2;
        y := (!y) + (!x);
        !y
\end{lstlisting}

\answerboxbig

\section*{Problem2 - 30\%}

Extend the $while$ loop covered in the lecture to a $do-while$ loop. Include both operational and typing semantics.

\answerboxbig

\section*{Problem3 - 40\%}

In lecture \emph{Going Imperative}, the language is extended with exceptions.
You can raise an exception by using \texttt{raise e} and trap it by using
the \texttt{try\dots with\dots} syntax.
In this problem, you are required to define the syntax and the semantics
(including evaluation rules and typing rules) of \texttt{try $e$ catch $e_1$ finally $e_2$}
in a way that is similar to how \texttt{finally} works in Java.
You can reuse some extensions such as sequence (\texttt{$e_1$;$e_2$}).

\answerboxbig

\end{document}