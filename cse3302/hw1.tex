%----PrefaceImport----%
\documentclass{article}
\usepackage{fancyhdr}
\usepackage[a4paper,margin=1in,headsep=25pt]{geometry}
\usepackage{lipsum,hyperref}
\usepackage{enumerate,fullpage,proof}
\usepackage[fontsize=12pt]{fontsize}
\usepackage{amsmath,amscd,amsbsy,amssymb,latexsym,url,bm,amsthm}
\usepackage{epsfig,graphicx,subfigure}
\usepackage[usenames]{xcolor}

\pagestyle{fancy}
\pagenumbering{Alph}
\setlength{\headheight}{36.0pt}
\headsep = 25pt
\fancyhf{}
\lhead{CSE 3302/5307 Programming Language Concepts}
\rhead{Homework1:Basic Concepts}
\lfoot{2024 Kenny Zhu & Essam Abdelghany}
\rfoot{Plagiarism will not be tolerated. We are here to help.}

%----Documentation----%
\begin{document}

\title{CSE 3302/5307 Programming Language Concepts}
\author{Homework 1 - Fall 2024}
\date{Due Date: Aug.26, 2024, 11:59p.m. Central Time}
\maketitle
\thispagestyle{fancy}
%----Homeworks----%
\section*{Problem 1 - 15\%}
{ Give a feature of C, C++ or Java that illustrates \textit{orthogonality}. Illustrate with justification whether there is a trade-off in efficiency due to orthogonality in this case.}


\section*{Problem 2 - 60\%}
Write a simple Lexer in Java that processes a computational statement given as a command line argument where the distinct tokens are separated by spaces. The Lexer will analyze each token and return an array that categorizes the type of each token. The types are returned in the same order as the corresponding tokens in the statement.
	
Sample output of main function:

\fbox{%

  \parbox{1\textwidth}{
Input statement: x + 10 - ( z )
                                                                   
Returned array of strings: [Identifier, Operator, Literal, Operator, Separator, Identifier, Separator]

  }

}

\begin{itemize}
    \item For operators, it's sufficient to consider \{+, -, *\}
    \item For identifiers, the token must be composed entirely of letters
    \item For literals, the token must be composed entirely of digits
    \item For separators, it's sufficient to consider \{(, ), ;\}
    \item \textbf{5\% Bonus within assignment:} use exception handling to throw a friendly error when a token is found to be none of the four types.

\end{itemize}
Please submit your `Lexer.java` file. Running:

\begin{verbatim}
javac Lexer.java
java Lexer "x + 10 - ( z )"
\end{verbatim}
Should output the types of each token, with each type printed on a new line.


\section*{Problem 3 - 25\%}
{ Compile a list of five compiled languages and five interpreted languages. Mention two advantages of each group of languages over the other. Explain at least one language translation technique that could be considered a middle ground between both.}



\end{document}
