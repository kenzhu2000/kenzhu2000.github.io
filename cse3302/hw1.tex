%----PrefaceImport----%
\documentclass{article}
\usepackage{fancyhdr}
\usepackage[a4paper,margin=1in,headsep=25pt]{geometry}
\usepackage{lipsum,hyperref}
\usepackage{enumerate,fullpage,proof}
\usepackage[fontsize=12pt]{fontsize}
\usepackage{amsmath,amscd,amsbsy,amssymb,latexsym,url,bm,amsthm}
\usepackage{epsfig,graphicx,subfigure}
\usepackage[usenames]{xcolor}

\pagestyle{fancy}
\pagenumbering{Alph}
\setlength{\headheight}{36.0pt}
\headsep = 25pt
\fancyhf{}
\lhead{CSE 3302/5307 Programming Language Concepts}
\rhead{Homework 1:Basic Concepts}
\lfoot{2025 Kenny Zhu}
%----Documentation----%
\begin{document}

\title{CSE 3302/5307 Programming Language Concepts}
\author{Homework 1 - Fall 2025}
\date{Due Date: Aug 25, 2025, 8:00PM CT}
\maketitle
\thispagestyle{fancy}
%----Homeworks----%
\section*{Problem 1 - 20\%}
{ Give a feature of C, C++ or Java that illustrates \textit{orthogonality}. Give a feature that illustrates \textit{non-orthogonality}.}

\section*{Problem 2 - 20\%}
{ Choose a programming language you are familiar with. List its basic data types with explanation.}

\section*{Problem 3 - 25\%}
Write a Java function called \textit{SpOdd}. The function takes an array of integers as input and return an array of integers with all odd numbers in original array. Keep the same order as the original array. Then test your function in main function.
	
Sample output of main function:

\fbox
{ We have learned the difference between compiler and interpreter. Now research compiled languages and interpreted languages. Then list the \textit{advantages} and \textit{disadvantages} of these two types of languages.}


\end{document}
