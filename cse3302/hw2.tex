%----PrefaceImport----%
\documentclass{article}
\usepackage{fancyhdr}
\usepackage[a4paper,margin=1in,headsep=25pt]{geometry}
\usepackage{lipsum,hyperref}
\usepackage{enumerate,fullpage,proof}
\usepackage[fontsize=12pt]{fontsize}
\usepackage{amsmath,amscd,amsbsy,amssymb,latexsym,url,bm,amsthm}
\usepackage{epsfig,graphicx,subfigure}
\usepackage[usenames]{xcolor}

\pagestyle{fancy}
\pagenumbering{Alph}
\setlength{\headheight}{36.0pt}
\headsep = 25pt
\fancyhf{}
\lhead{CSE 3302/5307 Programming Language Concepts}
\rhead{Homework 2: Inductive Proofs}
\lfoot{2025 Kenny Zhu Wonjun Park}
\rfoot{Plagiarism will not be tolerated. We are here to help.}

% New command for blank spaces after questions
\newcommand{\answerbox}{
    \vspace{7cm} % Adjust the space size as needed
}
\newcommand{\answerboxbig}{
    \vspace{20cm} % Adjust the space size as needed
}
\newcommand{\answerboxsmall}{
    \vspace{3cm} % Adjust the space size as needed
}

% New command for personal info at the end of the document
\newcommand{\studentinfo}{
    \noindent Name: \underline{\hspace{5cm}} UTA ID: \underline{\hspace{5cm}}\\
    \vspace{0.5cm} % Space after the fields
}

%----Documentation----%
\begin{document}

\title{CSE 3302/5307 Programming Language Concepts}
\author{Homework 2 - Fall 2025}
\date{Due Date: Sep. 1, 2025, 8:00PM Central Time}
\maketitle
\thispagestyle{fancy}
%----Homeworks----%

\section*{Problem1 - 30\%}
\fontsize{12pt}{0}
\begin{enumerate}[(a)]    
    \item Consider looking at page 21 in slide "inductive-proof". In the proof of the second case $\frac{n\ nat}{S(n)\ nat}$, what is the assumption in this case and what is the difference between assumption and I.H.?
    \answerboxsmall

    \item We define a judgment form $IsNat\ x\ a$.
    \[
    \frac{x\ nat}{IsNat\ x\ true}Nat
    \qquad
    \frac{x\ list}{IsNat\ x\ false}List
    \qquad
	\frac{x\ tree}{IsNat\ x\ false}Tree
    \]
    For which rule we can use its inversion rule? If there exists such rule, point it out and give an explanation. If no rules can be inverted, give an explanation.
    \answerboxsmall
\end{enumerate}

\section*{Problem2 - 35\%}

\begin{enumerate}[(a)]
    \item Give an inductive definition of the judgment form
    $\mbox{max}\ n_1\ n_2\ n_3$, which indicates the max number between $n_1$ and $n_2$ is $n_3$. 
    \newline
    \textbf{Hint}: think of how we defined $add$ by knowledge of $nat$. 
    \answerboxsmall

    \item Prove by induction:
    if $\mbox{max}\ n_1\ n_2\ n_3$, then  $\mbox{max}\ n_2\ n_1\ n_3$.
    \answerboxsmall
\end{enumerate}

\section*{Problem3 - 35\%}

Recall the definition of natural numbers by $n\ \mbox{nat}$ judgement taught in the lecture.

\begin{enumerate}[(a)]
    \item Give an inductive definition of the judgement form
    $\mbox{fib}\ n_1\ n_2$, which indicates the $n_1^{th}$ Fibonacci number is $n_2$.
    \answerboxsmall

    \item Give an inductive definition of the judgement form
    $\mbox{fibsum}\ n_1\ n_2$, which indicates the sum of the first $n_1$ Fibonacci numbers is $n_2$.
    \answerboxsmall

    \item Prove by induction:
    If $\mbox{fibsum}\ n\ m$ then $\mbox{fib}\
    \mbox{succ}(\mbox{succ}(n))\ \mbox{succ}(m)$, that is
    $$\sum_{i=1}^n F_i = F_{n+2} - 1.$$
    \answerboxsmall

\end{enumerate}

% Include personal info fields at the end of the document
\studentinfo

\end{document}
