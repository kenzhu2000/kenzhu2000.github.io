%----PrefaceImport----%
\documentclass{article}
\usepackage{fancyhdr}
\usepackage[a4paper,margin=1in,headsep=25pt]{geometry}
\usepackage{lipsum,hyperref}
\usepackage{enumerate,fullpage,proof}
\usepackage[fontsize=12pt]{fontsize}
\usepackage{amsmath,amscd,amsbsy,amssymb,latexsym,url,bm,amsthm}
\usepackage{epsfig,graphicx,subfigure}
\usepackage[usenames]{xcolor}

\pagestyle{fancy}
\pagenumbering{Alph}
\setlength{\headheight}{36.0pt}
\headsep = 25pt
\fancyhf{}
\lhead{CSE 3302/5307 Programming Language Concepts}
\rhead{Homework2:Inductive Proof}
\lfoot{2023 Kenny Zhu}
%----Documentation----%
\begin{document}

\title{CSE 3302/5307 Programming Language Concepts}
\author{Homework2 - Fall 2023}
\date{Due Date: Sep.4, 2023, 8:00p.m. Central Time}
\maketitle
\thispagestyle{fancy}

%----Homeworks----%

\section*{Problem1 - 30\%}
\fontsize{12pt}{0}
\begin{enumerate}[(a)]
	\item Please look at page 21 in slide "inductive-proof". In the proof of the second case $\frac{n\ nat}{S(n)\ nat}$, what is the assumption in this case and what is the difference between assumption and I.H.?
	
	\item We define a judgment form $IsNat\ x\ a$.
	\[
	\frac{x\ nat}{IsNat\ x\ true}NatRule
	\qquad
	\frac{x\ list}{IsNat\ x\ false}ListRule
	\qquad
	\frac{x\ tree}{IsNat\ x\ false}TreeRule
	\]
	For which rule we can use its inversion rule? If there exists such rule, point it out and give an explanation. If no rules can be inverted, give an explanation.
	
	\item We define a judgment form $add'\ n_{1}\ n_{2}\ n_{3}$ (another definition for addition):
	\[
	\frac{}{add'\ Z\ Z\ Z}add'Z
	\qquad
	\frac{add'\ n_{1}\ n_{2}\ n_{3}}{add'\ (S n_{1})\ n_{2}\ (S n_{3})}add'-l
	\qquad
	\frac{add'\ n_{1}\ n_{2}\ n_{3}}{add'\ n_{1}\ (S n_{2})\ (S n_{3})}add'-r
	\]
	For which rule we can use its inversion rule? If there exists such rule, point it out and give an explanation. If no rules can be inverted, give an explanation. 
	
\end{enumerate}

\section*{Problem2 - 20\%}

\begin{enumerate}[(a)]
    \item Give an inductive definition of the judgment form
    $\mbox{max}\ n_1\ n_2\ n_3$, which indicates the max number between $n_1$ and $n_2$ is $n_3$.
    \item Prove by induction:
    if $\mbox{max}\ n_1\ n_2\ n_3$, then  $\mbox{max}\ n_2\ n_1\ n_3$.
\end{enumerate}

\section*{Problem3 - 20\%}

\begin{enumerate}[(a)]
    \item Recall the definition of addition by $add\ n_{1}\ n_{2}\ n_{3}$ judgment taught in the lecture.
    \item (15 points) Prove by induction: If$\ add\ n_{1}\ n_{2}\ n_{3}$, then $add\ n_{2}\ n_{1}\ n_{3}$ (Commutative law of add).
    
    (\textbf{Hint}: You can begin with proof of this lemma: If$\ n\ nat$, then $add\ n\ Z\ n$.)
\end{enumerate}

\section*{Problem4 - 30\%}

Recall the definition of natural numbers by $n\ \mbox{nat}$ judgment taught in the lecture.
\begin{enumerate}[(a)]
    \item Give an inductive definition of the judgment form
    $\mbox{fib}\ n_1\ n_2$, which indicates the $n_1^{th}$ Fibonacci number is $n_2$.
    \item Give an inductive definition of the judgment form
    $\mbox{fibsum}\ n_1\ n_2$, which indicates the sum of the first $n_1$ Fibonacci
    numbers is $n_2$.
    \item Prove by induction:
    If $\mbox{fibsum}\ n\ m$ then $\mbox{fib}\
    \mbox{succ}(\mbox{succ}(n))\ \mbox{succ}(m)$, that is
    $$\sum_{i=1}^n F_i = F_{n+2} - 1.$$
\end{enumerate}

\end{document}