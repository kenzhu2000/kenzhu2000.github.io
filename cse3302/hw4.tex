%----PrefaceImport----%
\documentclass{article}
\usepackage{fancyhdr}
\usepackage[a4paper,margin=1in,headsep=25pt]{geometry}
\usepackage{lipsum,hyperref}
\usepackage{enumerate,fullpage,proof}
\usepackage[fontsize=12pt]{fontsize}
\usepackage{amsmath,amscd,amsbsy,amssymb,latexsym,url,bm,amsthm}
\usepackage{epsfig,graphicx,subfigure}
\usepackage[usenames]{xcolor}

\pagestyle{fancy}
\pagenumbering{Alph}
\setlength{\headheight}{36.0pt}
\headsep = 25pt
\fancyhf{}
\lhead{CSE 3302/5307 Programming Language Concepts}
\rhead{Homework4:Untyped Lambda Calculus \uppercase\expandafter{\romannumeral2}}
\lfoot{2023 Kenny Zhu}
%----Documentation----%
\begin{document}

\title{CSE 3302/5307 Programming Language Concepts}
\author{Homework4 - Fall 2023}
\date{Due Date: Sep.23, 2023, 8:00p.m. Central Time}
\maketitle
\thispagestyle{fancy}

%----Homeworks----%

\section*{Problem1 - 40\%}

Given the definition of \textit{pred n} (predecessor of n):
\[pred = \lambda n.\lambda f.\lambda x.n\ (\lambda g.\lambda h.h\ (g\ f))\ (\lambda u.x)\ (\lambda u.u)\]    

Please define following terms using lambda calculus:

\begin{enumerate}
    \item sub m n (subtraction)
    \item iszero n
    \item leq m n (m is less or equal than n)
    \item equal m n
    \item factorial n (hint: try to define it using pair)
\end{enumerate}
(You can directly use the definition in the slides and the last homework, like add, tru, etc.)


\section*{Problem2 - 20\%}

Prove the \textbf{exchange lemma}: If $\Gamma, x:t_1,y:t_2, \Gamma'\vdash e:t, $ then $\Gamma, y:t_2, x:t_1, \Gamma' \vdash e:t.$ (proof by induction on derivation of $\Gamma, x:t_1, y:t_2, \Gamma' \vdash e:t$).


\section*{Problem3 - 20\%}

Prove the \textbf{weakening lemma}: If $\Gamma \vdash e:t$ then $\Gamma,x:t' \vdash e:t$ (provided x not in Dom($\Gamma$)).


\section*{Problem4 - 30\%}

Prove the \textbf{substitution lemma}: If $\Gamma,x:t' \vdash e:t$ and $\Gamma \vdash v:t'$ then $\Gamma \vdash e[v/x]:t$.


\textbf{Remark:} 

Please email \textbf{.pdf} files to TA.

File name format: {\color{red} HW\_X\_FirstName\_10digitID.pdf}

Example: {HW\_3\_Sinong\_1001001000.pdf}

\end{document}