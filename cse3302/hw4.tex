%----PrefaceImport----%
\documentclass{article}
\usepackage{fancyhdr}
\usepackage[a4paper,margin=1in,headsep=25pt]{geometry}
\usepackage{lipsum,hyperref}
\usepackage{enumerate,fullpage,proof}
\usepackage[fontsize=12pt]{fontsize}
\usepackage{amsmath,amscd,amsbsy,amssymb,latexsym,url,bm,amsthm}
\usepackage{epsfig,graphicx,subfigure}
\usepackage[usenames]{xcolor}

\pagestyle{fancy}
\pagenumbering{Alph}
\setlength{\headheight}{36.0pt}
\headsep = 25pt
\fancyhf{}
\lhead{CSE 3302/5307 Programming Language Concepts}
\rhead{Homework 4: Lambda}
\lfoot{2025 Kenny Zhu Wonjun Park}
\rfoot{Plagiarism will not be tolerated. We are here to help.}

% New command for blank spaces after questions
\newcommand{\answerbox}{
    \vspace{7cm} % Adjust the space size as needed
}
\newcommand{\answerboxbig}{
    \vspace{20cm} % Adjust the space size as needed
}
\newcommand{\answerboxsmall}{
    \vspace{3cm} % Adjust the space size as needed
}

% New command for personal info at the end of the document
\newcommand{\studentinfo}{
    $$\begin{array}{cc}
        \noindent \text{Name:} \underline{\hspace{5cm}} &
            \text{UTA ID:} \underline{\hspace{5cm}}\\
    \end{array}$$
}

%----Documentation----%
\begin{document}

\title{CSE 3302/5307 Programming Language Concepts}
\author{Homework 4 - Fall 2025}
\date{Due Date: Sep. 15, 2025, 9:00PM Central Time}
\maketitle
\thispagestyle{fancy}
%----Homeworks----%

% Include personal info fields at the end of the document
\studentinfo

\section*{Problem1 - 40\%}

	Evaluate the following $\lambda$ expressions using call-by-value and call-by-name. Show the complete steps of evaluation.
	\begin{enumerate}[(a)]
		\item $((\lambda z.((\lambda x.\ x-y+z)\ 3))\ 2)$
		\item $((\lambda v.(\lambda w.w))\ ((\lambda x.x)\ (y\ (\lambda z.z))))$
		\item $((\lambda x.\ x\ x)\ (\lambda y.\ y\ y))$
		\item $((\lambda x. \lambda y. x)\ (\lambda z.z\ \lambda u.u))$
	\end{enumerate}

\answerboxbig

\section*{Problem2 - 30\%}

    Prove by induction: If $FV(e_{1})=\emptyset$ and $e_{1} \rightarrow e_{2}$, then $FV(e_{2})=\emptyset$.
    \begin{itemize}
    \item Given the following definitions:
    \begin{enumerate}
        \item Rules of free variables 
        \[
        \frac{}{FV(x)=\{x\}} \qquad
        \frac{FV(e_1) = S_1\quad FV(e_2) = S_2}{FV(e_1\ e_2)=S_1\cup S_2} \qquad
        \frac{FV(e) = S}{FV(\\x.e)=S-\{x\}}
        \]
        \item Judgment form: \textbf{define} $e_1\rightarrow e_2$
        \[
        \frac{}{(\lambda x.e)\ v \rightarrow e [v/x]}
        \qquad
        \frac{e_1 \rightarrow e_1^{'}}{e_1\ e_2 \rightarrow e_1^{'}\ e_2}
        \qquad
        \frac{e_2 \rightarrow e_2^{'}}{v\ e_2 \rightarrow v\ e_2^{'}}
        \]
    \end{enumerate}
    \item And given this lemma:
        \begin{lemma}\label{fv}
            $FV(e_1[e_2/x]) \subseteq (FV(e_1) - \{x\}) \cup FV(e_2)$
        \end{lemma}
    \end{itemize}

\vspace{20pt}
\textbf{By induction on derivation} of $e_1\rightarrow e_2$

1. Case $\frac{}{(\lambda x.e)\ v \rightarrow e [v/x]}$
Need to Prove:
\answerbox

2. Case $\frac{e_1 \rightarrow e_1^{'}}{e_1\ e_2 \rightarrow e_1^{'}\ e_2}$
Need to Prove:
\answerbox

3. Case $\frac{e_2 \rightarrow e_2^{'}}{v\ e_2 \rightarrow v\ e_2^{'}}$
Need to Prove:
\answerbox


\section*{Problem3 - 30\%}
Church numerals use lambdas to create a representation of numbers.
They can represent natural numbers $\mathbf{0}, \mathbf{1}, \mathbf{2}, ...,$ as follows:
\begin{align*}
    \mathbf{0} &= \lambda f.\lambda x.\ x \\
    \mathbf{1} &= \lambda f.\lambda x.\ f\ x \\
    \mathbf{2} &= \lambda f.\lambda x.\ f\ (f\ x) \\
    \mathbf{3} &= \lambda f.\lambda x.\ f\ (f\ (f\ x)) \\
    &\dots \\
    \mathbf{n} &= \lambda f.\lambda x.\ f^n\ x \\
    &\dots
\end{align*}
Church numerals takes two parameters $f$ and $x$. Church numerals $n$ means apply $f$ to $x$ $n$ times. You can read more about church numerals on the internet.
\begin{enumerate}[(a)]
    \item Define addition in $\lambda$ calculus, and then show the evaluation of $3+7$.
    \answerbox
    \item Define multiplication in $\lambda$ calculus (Hint: you can use definition of addition), and then show the evaluation of $6\times2$.
    \answerbox

\end{enumerate}

\vspace{20pt}
\end{document}