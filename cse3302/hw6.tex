%----PrefaceImport----%
\documentclass{article}
\usepackage{fancyhdr}
\usepackage[a4paper,margin=1in,headsep=25pt]{geometry}
\usepackage{lipsum,hyperref}
\usepackage{enumerate,fullpage,proof}
\usepackage[fontsize=12pt]{fontsize}
\usepackage{amsmath,amscd,amsbsy,amssymb,latexsym,url,bm,amsthm}
\usepackage{epsfig,graphicx,subfigure}
\usepackage[usenames]{xcolor}
\usepackage{tcolorbox}
\usepackage{amsmath}

\newtheorem{thm}{Theorem}
\newtheorem{lemma}[thm]{Lemma}

\pagestyle{fancy}
\pagenumbering{Alph}
\setlength{\headheight}{36.0pt}
\headsep = 25pt
\fancyhf{}
\lhead{CSE 3302/5307 Programming Language Concepts}
\rhead{Homework 6: Simply-typed Extensions II}
\lfoot{2024 Kenny Zhu & Essam Abdelghany}
\rfoot{Plagiarism will not be tolerated. We are here to help.}

% New command for blank spaces after questions
\newcommand{\answerbox}{
    \vspace{7cm} % Adjust the space size as needed
}
\newcommand{\answerboxbig}{
    \vspace{20cm} % Adjust the space size as needed
}
\newcommand{\answerboxsmall}{
    \vspace{3cm} % Adjust the space size as needed
}

% New command for personal info at the end of the document
\newcommand{\studentinfo}{
    \noindent Name: \underline{\hspace{5cm}} UTA ID: \underline{\hspace{5cm}}\\
    \vspace{0.5cm} % Space after the fields
}

\usepackage{listings}
\usepackage{xcolor}

\lstset{
    language=C,
    basicstyle=\ttfamily\footnotesize,
    keywordstyle=\color{blue},
    commentstyle=\color{gray},
    stringstyle=\color{orange},
    numbers=none, % Hides line numbers
    showstringspaces=false,
    breaklines=true,
    frame=single,
    rulecolor=\color{black},
}


%----Documentation----%
\begin{document}

\title{CSE 3302/5307 Programming Language Concepts}
\author{Homework6 - Fall 2023}
\date{Due Date: Sep.30, 2024, 11:59p.m. Central Time}
\maketitle
\thispagestyle{fancy}

%----Homeworks----%

\section*{Problem1 - 30\%}

	We've seen how to define natural numbers using church encoding in untyped lambda calculus:
	\begin{align*}
		\mathbf{0} &= \lambda f.\lambda x.\ x \\
		\mathbf{1} &= \lambda f.\lambda x.\ f\ x \\
		&\dots \\
		\mathbf{n} &= \lambda f.\lambda x.\ f^n\ x \\
		&\dots
	\end{align*}
	Note that church encoding cannot represent negative integers.
	\begin{enumerate}[(a)]
	\item Propose a \textbf{simple} method to extend church numerals to representation of integers. Give a concrete example for representation of integer \textbf{-5} with your proposed method. Hint: you may try to use pairs.
	
	\item Define the XOR function given two boolean inputs in lambda calculus and test that it works. 

 \item Define a new multiplication operation $mulint$ that works on the representation of integers you defined.
\end{enumerate}

 \noindent For the last two points, you can directly use basic logical functions defined in the lecture such as $not$, $and$ and $or$ as well as the $mul$ you wrote for natural numbers in an earlier assignment as well as .


\newpage 
\ % The empty page
\newpage 


\section*{Problem2 - 30\%}

Given the definition of Fibonacci number
\[F_0 = 0, F_1 = 1, F_i = F_{i-1} + F_{i-2}\]
\begin{enumerate}[(a)]
\item Use \textit{fix} to write a lambda function called \textit{fib}: int $\rightarrow$ int to compute the n-th Fibonacci number.
\item Test that your function works by showing detailed steps for $fib \, 3$. Use the Z combinator defined in the lecture for $fix$ and do not treat it as a black box.
\end{enumerate}
\answerboxbig

\section*{Problem3 - 40\%}

	Given the following $\lambda$ expression:
 
	\begin{verbatim}
	let x = 2 in
	  let y = 4 in
	    let f1 = \x.\y.x+2*y in
	      let f2 = \x.\y.2*x-y in
	      f2 (f1 y x) 3
	\end{verbatim}


	
	Using the environment model for lambda calculus with let,
		
	(a) Define closures. (Be careful and refer to lecture slides);
	 
	(b) Show detailed multi-step evaluation process of the $\lambda$ expression above. 
 
 
 The environment should be clearly shown in each step.

\section*{}
\answerboxbig

\studentinfo


\end{document}