%----PrefaceImport----%
\documentclass{article}
\usepackage{fancyhdr}
\usepackage[a4paper,margin=1in,headsep=25pt]{geometry}
\usepackage{lipsum,hyperref}
\usepackage{enumerate,fullpage,proof}
\usepackage[fontsize=12pt]{fontsize}
\usepackage{amsmath,amscd,amsbsy,amssymb,latexsym,url,bm,amsthm}
\usepackage{epsfig,graphicx,subfigure}
\usepackage[usenames]{xcolor}

\pagestyle{fancy}
\pagenumbering{Alph}
\setlength{\headheight}{36.0pt}
\headsep = 25pt
\fancyhf{}
\lhead{CSE 3302/5307 Programming Language Concepts}
\rhead{Homework6:Extentions to Sim-Typed LC\uppercase\expandafter{\romannumeral2}}
\lfoot{2023 Kenny Zhu}
%----Documentation----%
\begin{document}

\title{CSE 3302/5307 Programming Language Concepts}
\author{Homework6 - Fall 2023}
\date{Due Date: Oct.7, 2023, 8:00p.m. Central Time}
\maketitle
\thispagestyle{fancy}

%----Homeworks----%

\section*{Problem1 - 40\%}

We've seen how to define natural numbers using church encoding in untyped lambda calculus:
\begin{align*}
    \mathbf{0} &= \lambda f.\lambda x.\ x \\
    \mathbf{1} &= \lambda f.\lambda x.\ f\ x \\
    &\dots \\
    \mathbf{n} &= \lambda f.\lambda x.\ f^n\ x \\
    &\dots
\end{align*}
Note that church encoding cannot represent negative integers, we try to encode all integers using \textbf{untyped} lambda calculus.
\begin{enumerate}[(a)]
\item Propose a method to extend church numerals to representation of integers.(Hint: you may try to use pairs). Give a concrete example for representation of integer \textbf{-5} with your proposed method.

\item Define a function $nat2int$ that converts a natural number to your representation of correspondent integer.

\item Based on this definition of integers, define the following arithmetic operations in lambda calculus(you can directly use operations on natural numbers defined before like add, multi, etc. ): 
    \begin{enumerate}[(1)]
        \item negation: neg n
        \item addition: addint m n
        \item subtraction: subint m n
        \item multiplication: multint m n	
    \end{enumerate}
\item Bonus: Are there other ways to implement integers? Explain your idea briefly with some example for operations.
\end{enumerate}

\section*{Problem2 - 30\%}

Given the definition of Fibonacci number
\[F_0 = 0, F_1 = 1, F_i = F_{i-1} + F_{i-2}\]
\begin{enumerate}[(a)]
\item Use \textit{fix} to write a lambda function called \textit{fib}: int $\rightarrow$ int to compute the n-th Fibonacci number.
\item We want to extend simple \textit{let} expression to recursive \textit{let rec} expression:
\[letrec\ f = \lambda x.\ e_{1}\ in\ e_{2}\]
where f itself can appear in $e_{1}$. 

Example usage of \textit{letrec} for factorial:
\[fact = \lambda n. (letrec\ fact= (\lambda i.\ if\ i=0\ then\ 1\ else\ i*(fact\ (i-1))) in\ fact\ n)\]

\begin{enumerate}[(1)]
\item Define semantic and typing rules for expression \textit{letrec} ;
\item Use \textit{letrec} to redefine our Fibonacci function.

\end{enumerate}

\end{enumerate}

\section*{Problem3 - 30\%}

Given the following $\lambda$ expression:
\begin{verbatim}
let x = 2 in
  let y = 4 in
    let f1 = \x.\y.x+2*y in
      let f2 = \x.\y.2*x-y in
      f2 (f1 y x) 3
\end{verbatim}

Using the environment model for lambda calculus with let,
    
(a) Define closures. (Be careful and refer to lecture slides);
 
(b) Show detailed multi-step evaluation process of the $\lambda$ expression above.




\end{document}