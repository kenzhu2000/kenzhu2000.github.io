%----PrefaceImport----%
\documentclass{article}
\usepackage{fancyhdr}
\usepackage[a4paper,margin=1in,headsep=25pt]{geometry}
\usepackage{lipsum,hyperref}
\usepackage{enumerate,fullpage,proof}
\usepackage[fontsize=12pt]{fontsize}
\usepackage{amsmath,amscd,amsbsy,amssymb,latexsym,url,bm,amsthm}
\usepackage{epsfig,graphicx,subfigure}
\usepackage[usenames]{xcolor}
\usepackage{tcolorbox}

\newtheorem{thm}{Theorem}
\newtheorem{lemma}[thm]{Lemma}

\pagestyle{fancy}
\pagenumbering{Alph}
\setlength{\headheight}{36.0pt}
\headsep = 25pt
\fancyhf{}
\lhead{CSE 3302/5307 Programming Language Concepts}
\rhead{Homework 6: Simply-typed Extensions}
\lfoot{2025 Kenny Zhu Wonjun Park}
\rfoot{Plagiarism will not be tolerated. We are here to help.}

% New command for blank spaces after questions
\newcommand{\answerbox}{
    \vspace{7cm} % Adjust the space size as needed
}
\newcommand{\answerboxbig}{
    \vspace{20cm} % Adjust the space size as needed
}
\newcommand{\answerboxsmall}{
    \vspace{3cm} % Adjust the space size as needed
}

% New command for personal info at the end of the document
\newcommand{\studentinfo}{
    $$\begin{array}{cc}
        \noindent \text{Name:} \underline{\hspace{5cm}} &
            \text{UTA ID:} \underline{\hspace{5cm}}\\
    \end{array}$$
}

\usepackage{listings}
\usepackage{xcolor}

\lstset{
    language=Java,
    basicstyle=\ttfamily\footnotesize,
    keywordstyle=\color{blue},
    commentstyle=\color{gray},
    stringstyle=\color{orange},
    numbers=none, % Hides line numbers
    showstringspaces=false,
    breaklines=true,
    frame=single,
    rulecolor=\color{black},
}


%----Documentation----%
\begin{document}

\title{CSE 3302/5307 Programming Language Concepts}
\author{Homework 6 - Fall 2025}
\date{Due Date: Sep. 29, 2025, 9:00PM Central Time}
\maketitle
\thispagestyle{fancy}

\studentinfo

%----Homeworks----%

\section*{Problem1 - 30\%}

Consider the following program which is written in Java syntax.
\begin{lstlisting}
public class test {
    int x = 3;

    public void f1() {
        int x = 10;
        f2();
        System.out.println(x);
    }

    public void f2() {
        System.out.println(x);
    }

    public static void main(String[] args) {
        test a = new test();
        a.f1();
    }
}
\end{lstlisting}
What will be printed after running \texttt{main()} when it uses static scoping and dynamic scoping, respectively? Justify the operation in each case.

\answerboxbig

\section*{Problem2 - 40\%}

Extend tuples to records, and write the (a) syntax for expression and value form(s) in BNF and (b) operational semantic rules for records.
Example usage for records:
\begin{itemize}
    \item Elements are indexed by labels:
    \begin{itemize}
        \item $\{y=10\}$
        \item $\{id=1,salary=50000,active=\mathbf{true}\}$
    \end{itemize}
    \item The order of the record fields is insignificant:
    \begin{itemize}
        \item $\{y=10,x=5\}$ is the same as $\{x=5,y=10\}$
    \end{itemize}
    \item To access fields of a record:
    \begin{itemize}
        \item $a.id$
        \item $b.salary$
    \end{itemize}
\end{itemize}
Including typing semantics and their syntax correctly would correspond to a 5\% bonus for the previous assignment.
\answerboxbig

\section*{Problem3 - 30\%}

For the syntax of pairs and tuples, let's change(or add) $e.1$ $e.2$ or $e.i$ to $e_1.e_2$ so that the previous static projection is replaced by the dynamic projection, meaning that $e_2$ can be any Lambda expression that evaluates to an
index value. 
The base types are $bool\ |\ int\ |\ t1\ \rightarrow\ t2$. Update the typing rules and semantic rules (still left-to-right call-by-value operational semantics)
for pairs and tuples accordingly, checking the type of $e_2$ in $e_1.e_2$ and the value of $e_2$ for not going out of bounds.

\end{document}
