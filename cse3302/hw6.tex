%----PrefaceImport----%
\documentclass{article}
\usepackage{fancyhdr}
\usepackage[a4paper,margin=1in,headsep=25pt]{geometry}
\usepackage{lipsum,hyperref}
\usepackage{enumerate,fullpage,proof}
\usepackage[fontsize=12pt]{fontsize}
\usepackage{amsmath,amscd,amsbsy,amssymb,latexsym,url,bm,amsthm}
\usepackage{epsfig,graphicx,subfigure}
\usepackage[usenames]{xcolor}

\pagestyle{fancy}
\pagenumbering{Alph}
\setlength{\headheight}{36.0pt}
\headsep = 25pt
\fancyhf{}
\lhead{CSE 3302}
\rhead{Homework 6:Extentions to Sim-Typed LC\uppercase\expandafter{\romannumeral2}}
\lfoot{2026 Kenny Zhu}
%----Documentation----%
\begin{document}

\title{CSE 3302 Programming Language}
\author{Homework 6 - Fall 2026}
\date{Due Date: March 2, 2026, 11:59 PM}
\maketitle
\thispagestyle{fancy}

%----Homeworks----%


\section*{Problem 1 - 40\%}

Given the definition of Fibonacci number
\[F_0 = 0, F_1 = 1, F_i = F_{i-1} + F_{i-2}\]
\begin{enumerate}[(a)]
\item Use \textit{fix} to write a lambda function called \textit{fib}: int $\rightarrow$ int to compute the n-th Fibonacci number.
\item We want to extend simple \textit{let} expression to recursive \textit{let rec} expression:
\[letrec\ f = \lambda x.\ e_{1}\ in\ e_{2}\]
where f itself can appear in $e_{1}$. 

Example usage of \textit{letrec} for factorial:
\[fact = \lambda n. (letrec\ fact= (\lambda i.\ if\ i=0\ then\ 1\ else\ i*(fact\ (i-1))) in\ fact\ n)\]

\begin{enumerate}[(1)]
\item Define semantic and typing rules for expression \textit{letrec} ;
\item Use \textit{letrec} to redefine our Fibonacci function.

\end{enumerate}

\end{enumerate}


\section*{Problem 2 - 30\%}

Refer to Slides 21. Provide the complete derivation tree of the following expression:
\begin{verbatim}
let x = 1 in
  let f = \y. y + x in
    let g = (\x. f x) + 1 in
      g (f x)
\end{verbatim}

\vspace{2.8em}

\section*{Problem 3 - 30\%}

Given the following $\lambda$ expression:
\begin{verbatim}
let x = 2 in
  let y = 4 in
    let f1 = \x.\y.x+2*y in
      let f2 = \x.\y.2*x-y in
      f2 (f1 y x) 3
\end{verbatim}

Using the environment model for lambda calculus with let,
    
(a) Define closures. (Be careful and refer to lecture slides);
 
(b) Show detailed multi-step evaluation process of the $\lambda$ expression above.

\vspace{20pt}
\vspace{0.8em}
\noindent{\color{red}\footnotesize
\textbf{Submission Format:}
Submit only the \texttt{.pdf} version of your homework (typed submissions are preferred;
Scanned images must be readable). File must be named
\texttt{lastname\_studentID\_hw5.pdf}.
}


\end{document}

