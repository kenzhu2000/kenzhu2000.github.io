%----PrefaceImport----%
\documentclass{article}
\usepackage{fancyhdr}
\usepackage[a4paper,margin=1in,headsep=25pt]{geometry}
\usepackage{lipsum,hyperref}
\usepackage{enumerate,fullpage,proof}
\usepackage[fontsize=12pt]{fontsize}
\usepackage{amsmath,amscd,amsbsy,amssymb,latexsym,url,bm,amsthm}
\usepackage{epsfig,graphicx,subfigure}
\usepackage[usenames]{xcolor}
\usepackage{tcolorbox}

\newtheorem{thm}{Theorem}
\newtheorem{lemma}[thm]{Lemma}

\pagestyle{fancy}
\pagenumbering{Alph}
\setlength{\headheight}{36.0pt}
\headsep = 25pt
\fancyhf{}
\lhead{CSE 3302/5307 Programming Language Concepts}
\rhead{Homework 5: Simply-typed Extensions}
\lfoot{2024 Kenny Zhu & Essam Abdelghany}
\rfoot{Plagiarism will not be tolerated. We are here to help.}

% New command for blank spaces after questions
\newcommand{\answerbox}{
    \vspace{7cm} % Adjust the space size as needed
}
\newcommand{\answerboxbig}{
    \vspace{20cm} % Adjust the space size as needed
}
\newcommand{\answerboxsmall}{
    \vspace{3cm} % Adjust the space size as needed
}

% New command for personal info at the end of the document
\newcommand{\studentinfo}{
    \noindent Name: \underline{\hspace{5cm}} UTA ID: \underline{\hspace{5cm}}\\
    \vspace{0.5cm} % Space after the fields
}

\usepackage{listings}
\usepackage{xcolor}

\lstset{
    language=C,
    basicstyle=\ttfamily\footnotesize,
    keywordstyle=\color{blue},
    commentstyle=\color{gray},
    stringstyle=\color{orange},
    numbers=none, % Hides line numbers
    showstringspaces=false,
    breaklines=true,
    frame=single,
    rulecolor=\color{black},
}


%----Documentation----%
\begin{document}

\title{CSE 3302/5307 Programming Language Concepts}
\author{Homework5 - Fall 2023}
\date{Due Date: Sep.30, 2024, 11:59p.m. Central Time}
\maketitle
\thispagestyle{fancy}

%----Homeworks----%

\section*{Problem1 - 30\%}

Consider the following program which is written in C syntax.
\begin{lstlisting}
int x = 1;

void f1() {
    int x = 3;
    f2();
    printf(x)
}

void f2() {
    int x = 2;
    printf(x)
}

int main() {
    f1();
    printf(x)
}
\end{lstlisting}
What will be printed after running \texttt{main()} when it uses static scoping?
    dynamic scoping? Justify the operation in each case.
\answerboxbig

\section*{Problem2 - 40\%}

Extend tuples to records, and write the (a) syntax for expression and value form(s) in BNF and (b) operational semantic rules for records. You can assume $x_1$,$x_2$,...,$x_n$ for arbitrary attribute names (and $x_j$ for an arbitrary attribute).
Example usage for records:
\begin{itemize}
    \item Elements are indexed by labels:
    \begin{itemize}
        \item $\{y=10\}$
        \item $\{id=1,salary=50000,active=\mathbf{true}\}$
    \end{itemize}
    \item The order of the record fields is insignificant:
    \begin{itemize}
        \item $\{y=10,x=5\}$ is the same as $\{x=5,y=10\}$
    \end{itemize}
    \item To access fields of a record:
    \begin{itemize}
        \item $a.id$
        \item $b.salary$
    \end{itemize}
\end{itemize}
Including typing semantics and their syntax correctly would correspond to a 5\% bonus for the previous assignment.
\answerboxbig

\section*{Problem3 - 30\%}

Another way of defining \textbf{let} might be to desugar it by executing it immediately. That is, to regard \textbf{Let x=$t_1$ in $t_2$} as an abbreviation for the substituted body \textbf{$t_2[t_1/x]$}. Consider implementing a functional language with call-by-value evaluation, would this be a good idea? Justify why or why not. Explictly state any conditions for the equivalence of \textbf{$t_2[t_1/x]$} and \textbf{Let x=$t_1$ in $t_2$} in your justification.

\section*{}
\answerboxbig

\studentinfo


\end{document}