%----PrefaceImport----%
\documentclass{article}
\usepackage{fancyhdr}
\usepackage[a4paper,margin=1in,headsep=25pt]{geometry}
\usepackage{lipsum,hyperref}
\usepackage{enumerate,fullpage,proof}
\usepackage[fontsize=12pt]{fontsize}
\usepackage{amsmath,amscd,amsbsy,amssymb,latexsym,url,bm,amsthm}
\usepackage{epsfig,graphicx,subfigure}
\usepackage{listings}
\usepackage[usenames]{xcolor}
\usepackage{tcolorbox}
\usepackage{amsmath}
\usepackage{listings}
\usepackage{xcolor}
\usepackage{enumerate,fullpage,proof}

\newtheorem{thm}{Theorem}
\newtheorem{lemma}[thm]{Lemma}
\newenvironment{sol}
  {\par\vspace{3mm}\noindent{\it Solution}.}
  {\qed}

\pagestyle{fancy}
\pagenumbering{Alph}
\setlength{\headheight}{36.0pt}
\headsep = 25pt
\fancyhf{}
\lhead{CSE 3302/5307 Programming Language Concepts}
\rhead{Homework 12: Subtyping}
\lfoot{2025 Kenny Zhu Wonjun Park}
\rfoot{Plagiarism will not be tolerated. We are here to help.}

\lstset{
  language=[Objective]Caml,
  basicstyle=\ttfamily,
  keywordstyle=\color{blue},
  commentstyle=\color{gray},
  stringstyle=\color{green},
  showstringspaces=false,
  numbers=left,
  numberstyle=\tiny,
  breaklines=true,
  frame=single
}

% New command for blank spaces after questions
\newcommand{\answerbox}{
    \vspace{7cm} % Adjust the space size as needed
}
\newcommand{\answerboxbig}{
    \vspace{20cm} % Adjust the space size as needed
}
\newcommand{\answerboxsmall}{
    \vspace{3cm} % Adjust the space size as needed
}

\newcommand{\studentinfo}{
    $$\begin{array}{cc}
        \noindent \text{Name:} \underline{\hspace{5cm}} &
            \text{UTA ID:} \underline{\hspace{5cm}}\\
    \end{array}$$
}

\usepackage{listings}
\usepackage{xcolor}
\usepackage{placeins}

\lstset{
    language=C,
    basicstyle=\ttfamily\footnotesize,
    keywordstyle=\color{blue},
    commentstyle=\color{gray},
    stringstyle=\color{orange},
    numbers=none, % Hides line numbers
    showstringspaces=false,
    breaklines=true,
    frame=single,
    rulecolor=\color{black},
}


\begin{document}

\title{CSE 3302/5307 Programming Language Concepts}
\author{Homework 12 - Fall 2025}
\date{Due Date: Nov. 10, 2025, 9:00PM Central Time}
\maketitle
\thispagestyle{fancy}

%----Homeworks----%

\section*{Problem 1 - 40\%}

Remember in hw5, we extent tuples to records. Now we extend subtypes to records. Please give some subtyping rules for record type, then draw a derivation showing that $\{x:Nat, y:Nat, z:Nat\}$ is a subtype of $\{y:Nat\}$.

\answerboxbig


\section*{Problem 2 - 60\%}

Prove Lemma [Inversion of the subtype relation]:
	
	1. If $S <= T_1 \rightarrow T_2$, then $S$ has the form $S_1 \rightarrow S_2$, with $T_1 <= S_1$ and $S_2 <= T_2$.
	
	2. If $S <= \{l_i:T_i^{i \in l...n}\}$, then $S$ has the form $\{k_j:S_j^{j \in l...m}\}$, with at least the labels $\{l_i^{i \in l...n}\}$ (i.e., $\{l_i^{i \in l...n}\} \subseteq \{k_j^{j \in l...m}\}$) and with $S_j <= T_i$ for each common label $l_i=k_j$.


\answerboxbig

\end{document}
