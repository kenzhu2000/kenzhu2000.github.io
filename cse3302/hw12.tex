%----PrefaceImport----%
\documentclass{article}
\usepackage{fancyhdr}
\usepackage[a4paper,margin=1in,headsep=25pt]{geometry}
\usepackage{lipsum,hyperref}
\usepackage{enumerate,fullpage,proof}
\usepackage[fontsize=12pt]{fontsize}
\usepackage{amsmath,amscd,amsbsy,amssymb,latexsym,url,bm,amsthm}
\usepackage{epsfig,graphicx,subfigure}
\usepackage{listings}
\usepackage[usenames]{xcolor}

\newtheorem{thm}{Theorem}
\newtheorem{lemma}[thm]{Lemma}

\pagestyle{fancy}
\pagenumbering{Alph}
\setlength{\headheight}{36.0pt}
\headsep = 25pt
\fancyhf{}
\lhead{CSE 3302/5307 Programming Language Concepts}
\rhead{Homework 12:Logic}
\lfoot{2024 Kenny Zhu & Essam Abdelghany}
\rfoot{Plagiarism will not be tolerated. We are here to help.}

\lstset{
  language=[Objective]Caml,
  basicstyle=\ttfamily,
  keywordstyle=\color{blue},
  commentstyle=\color{gray},
  stringstyle=\color{green},
  showstringspaces=false,
  numbers=left,
  numberstyle=\tiny,
  breaklines=true,
  frame=single
}

% New command for blank spaces after questions
\newcommand{\answerbox}{
    \vspace{7cm} % Adjust the space size as needed
}
\newcommand{\answerboxbig}{
    \vspace{20cm} % Adjust the space size as needed
}
\newcommand{\answerboxsmall}{
    \vspace{3cm} % Adjust the space size as needed
}

% New command for personal info at the end of the document
\newcommand{\studentinfo}{
    \noindent Name: \underline{\hspace{5cm}} UTA ID: \underline{\hspace{5cm}}\\
    \vspace{0.5cm} % Space after the fields
}

\usepackage{listings}
\usepackage{xcolor}
\usepackage{placeins}

\lstset{
    language=C,
    basicstyle=\ttfamily\footnotesize,
    keywordstyle=\color{blue},
    commentstyle=\color{gray},
    stringstyle=\color{orange},
    numbers=none, % Hides line numbers
    showstringspaces=false,
    breaklines=true,
    frame=single,
    rulecolor=\color{black},
}


\begin{document}

\title{CSE 3302/5307 Programming Language Concepts}
\author{Homework 12 - Fall 2024}
\date{Due Date: Nov.18, 2024, 11:59p.m. Central Time}
\maketitle
\thispagestyle{fancy}

%----Homeworks----%

\section*{Problem 1 - 40\%}

\begin{lstlisting}[language=Prolog]
bigger(elephant, horse).
bigger(horse, donkey).
bigger(donkey. dog).
bigger(donkey, monkey).
\end{lstlisting}

	We try to get familiar with the usage of SWI-Prolog and basic operations in this problem. Feel free to use other tools and follow the same steps.
	\begin{enumerate}[(a)]
		\item Download \href{http://www.swi-prolog.org/}{SWI-Prolog} and install (as well as the VSCode extension if needed).
		\item Consult the file \textit{animals.pl} (shown above) in SWI-Prolog. If there is an error, point out the line in which it occurs and fix it. 
		\item Re-consult the file. Enter the query as follows:
		
		\begin{verbatim}
			?- bigger(elephant, horse).
			?- bigger(elephant, monkey).
		\end{verbatim}
		\item Show the result of queries. For the second query, do we have the transitivity of bigger-relation as expected?
		\item Add rules called \textit{is\_bigger} to make sure the bigger-relation is transitive. An example output:
		\begin{verbatim}
			?- is_bigger(elephant, monkey).
			true
		\end{verbatim}
	\end{enumerate}
	\textbf{Remark:} 
    Submit one PDF report that includes the code and screenshot for the output.

\answerboxbig


\section*{Problem 2 - 60\%}

Read about the ELIZA chatbot and find a simple Prolog implementation \href{https://swish.swi-prolog.org/example/eliza.pl}{here}. Use comments to explain each rule in the code. Moreover, for each rule, write a unique query that would trigger the rule specifically (seven in total) and show the output. Extend the chatbot system so that it can answer one more type of question.

\textbf{Remark:} 
    Submit in the same PDF report as above the commented code and a screenshot for the output that shows (i), the output from invoking each of the seven queries that are meant to uniquely trigger a specific rule (should include the queries in the screenshots) and (ii), three queries for the eliza rule, one for each type of question and show their outputs.

\answerboxbig


\bibliographystyle{plain}
\end{document}
