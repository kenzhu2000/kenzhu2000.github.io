%----PrefaceImport----%
\documentclass{article}
\usepackage{fancyhdr}
\usepackage[a4paper,margin=1in,headsep=25pt]{geometry}
\usepackage{lipsum,hyperref}
\usepackage{enumerate,fullpage,proof}
\usepackage[fontsize=12pt]{fontsize}
\usepackage{amsmath,amscd,amsbsy,amssymb,latexsym,url,bm,amsthm}
\usepackage{epsfig,graphicx,subfigure}
\usepackage{listings}
\usepackage[usenames]{xcolor}

\newtheorem{thm}{Theorem}
\newtheorem{lemma}[thm]{Lemma}

\pagestyle{fancy}
\pagenumbering{Alph}
\setlength{\headheight}{36.0pt}
\headsep = 25pt
\fancyhf{}
\lhead{CSE 3302/5307 Programming Language Concepts}
\rhead{Homework9:Type Inference I}
\lfoot{2023 Kenny Zhu}
%----Documentation----%
\begin{document}

\title{CSE 3302/5307 Programming Language Concepts}
\author{Homework9 - Fall 2023}
\date{Due Date: Oct.28, 2023, 11:59p.m. Central Time}
\maketitle
\thispagestyle{fancy}

%----Homeworks----%

\section*{Problem1 - 60\%}

Given the following variant of untyped lambda calculus:
\newline
\begin{verbatim}
e::=
x (variables)
| c (constants)
| \x.e
| e1 e2
| e1 bop e2 (binary op)
| let x = e1 in e2
| if e1 then e2 else e3
| letfun f(x) = e1 in e2  (defining a recursive function f(x) for use in e2)
| inl e
| inr e
| case e1 of inl x => e2 | inr x => e3
| nil
| e1 :: e2
| case e1 of nil => e2 | x1 :: x2 => e3
| (e)
\end{verbatim}

\begin{enumerate}[(a)]
	\item Inductively define the constraint generation judgement:
	\begin{verbatim}
	G |- u ==> e:t, q
	\end{verbatim}
	
	\item Give the detailed derivation of the following expressions and obtain the set of equations, then solve these equations to get the principle solution and give the universal polymorphic types:
	\begin{enumerate}[(1)]
		\item
		\begin{verbatim}
		letfun sum(l) = case l of nil => 0 | x1 :: x2 => x1 + sum(x2) 
		in sum(12::10::0::nil)
		\end{verbatim}
		\item
		\begin{verbatim}
		let x = inr (5::4::3) in 
		case x of inl y => 0 | 
		          inr y => (case y of nil => 0 | h::l => h)
		\end{verbatim}
		\item 
		==BONUS POINT==
		\begin{verbatim}
		let x = inl {3::2::1, nil} in
		case x of inl y => (if y.2 == nil 
		then case y.1 of nil => 0 | h::l => h 
		else 0)
		inr y => (case y of nil => 0 | h::l => h)
		
		\end{verbatim}
		
	\end{enumerate}
\end{enumerate}


\section*{Problem2 - 40\%}

\begin{lemma}
    If a set of constraints q has a solution, then it has a most general one.
\end{lemma}

Prove this lemma.




\end{document}