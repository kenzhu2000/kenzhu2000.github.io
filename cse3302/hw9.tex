%----PrefaceImport----%
\documentclass{article}
\usepackage{fancyhdr}
\usepackage[a4paper,margin=1in,headsep=25pt]{geometry}
\usepackage{lipsum,hyperref}
\usepackage{enumerate,fullpage,proof}
\usepackage[fontsize=12pt]{fontsize}
\usepackage{amsmath,amscd,amsbsy,amssymb,latexsym,url,bm,amsthm}
\usepackage{epsfig,graphicx,subfigure}
\usepackage[usenames]{xcolor}
\usepackage{tcolorbox}
\usepackage{amsmath}
\usepackage{listings}
\usepackage{xcolor}
\usepackage{enumitem}   % For customizing the enumerate environment

\newtheorem{thm}{Theorem}
\newtheorem{lemma}[thm]{Lemma}

\pagestyle{fancy}
\pagenumbering{Alph}
\setlength{\headheight}{36.0pt}
\headsep = 25pt
\fancyhf{}
\lhead{CSE 3302/5307 Programming Language Concepts}
\rhead{Homework 9: Memory Management}
\lfoot{2025 Kenny Zhu Wonjun Park}
\rfoot{Plagiarism will not be tolerated. We are here to help.}

\lstset{
  language=[Objective]Caml,
  basicstyle=\ttfamily,
  keywordstyle=\color{blue},
  commentstyle=\color{gray},
  stringstyle=\color{green},
  showstringspaces=false,
  numbers=left,
  numberstyle=\tiny,
  breaklines=true,
  frame=single
}

% New command for blank spaces after questions
\newcommand{\answerbox}{
    \vspace{7cm} % Adjust the space size as needed
}
\newcommand{\answerboxbig}{
    \vspace{20cm} % Adjust the space size as needed
}
\newcommand{\answerboxsmall}{
    \vspace{3cm} % Adjust the space size as needed
}

\newcommand{\studentinfo}{
    $$\begin{array}{cc}
        \noindent \text{Name:} \underline{\hspace{5cm}} &
            \text{UTA ID:} \underline{\hspace{5cm}}\\
    \end{array}$$
}

\usepackage{listings}
\usepackage{xcolor}
\usepackage{placeins}

\lstset{
    language=C,
    basicstyle=\ttfamily\footnotesize,
    keywordstyle=\color{blue},
    commentstyle=\color{gray},
    stringstyle=\color{orange},
    numbers=none, % Hides line numbers
    showstringspaces=false,
    breaklines=true,
    frame=single,
    rulecolor=\color{black},
}


%----Documentation----%
\begin{document}

\title{CSE 3302/5307 Programming Language Concepts}
\author{Homework 9 - Fall 2025}
\date{Due Date: Oct. 20, 2025, 9:00PM Central Time}
\maketitle
\thispagestyle{fancy}

\studentinfo

%----Homeworks----%

\section*{Problem1 - 50\%}

\FloatBarrier % This ensures no floating objects (figures) appear before this point
\begin{figure}[h]
    \centering
    \includegraphics[width=0.95\textwidth]{Example.png}
    \caption{Heap Configuration}
    \label{fig:example}
\end{figure}

\noindent
Consider the heap configuration shown for allocated memory words. The heap is in total 100B in this case and one memory word is 4B.

\begin{enumerate}[label=\alph*)]
    \item Compute the amount of memory leaked for this configuration
    \item Mark the nodes that will be added to the free\_list after reference counting
    \item Compute the amount of memory leaked after each of reference counting and mark and sweep
    \item How does mark and sweep detect the cycle although it is not in the reference graph?
    \item Mention a set of conditions under which reference counting could be viewed to be significantly better than mark and sweep
\end{enumerate}

\newpage 
\mbox{}
\newpage


\section*{Problem2 - 20\%}

In Cheney's algorithm, the memory words in the from\_space are mapped in a consistent and contiguous sense to the to\_space. If that's the case, why is it necessary to store the forward addresses while mapping?

\answerboxbig


\section*{Problem3 - 30\%}

Here is a definition of the less-than-or-equal-to judgement for natural numbers:

Judgement Form:  $\vdash leq\ n_1\ n_2$

Rules:
\[
  \frac{n_2\ nat}{\vdash leq\ Z\ n_2}
  \tag{\sc Z-Leq}
\]
\[
  \frac{\vdash leq\ n_1\ n_2}{\vdash leq\ (S\ n_1)\ (S\ n_2)}
  \tag{\sc S-Leq}
\]

\begin{enumerate}[label=\alph*)]

    \item Use the $leq$ judgement to define a new judgement with the form
    \[ \vdash ascend\ l \]
    that is valid whenever the elements of $l$ are in ascending order
    (duplicates are allowed). For example, these judgements are valid:
    \[
    \vdash ascend\ cons (Z, cons (Z, cons (S\ S\ S\ Z, cons (S\ S\ S\ S\ S\ S\ Z, nil))))
    \]
    \[ \vdash ascend\ nil \]
    \[ \vdash ascend\ cons (S\ S\ Z, nil) \]
    This judgement is not valid:
    \[ \vdash ascend\ cons (Z, cons (S\ S\ Z, cons (S\ Z, nil))) \]

    \item Consider the judgement $\vdash dup\ l_1\ l_2$ and its rules:
    \[
    \frac{}{\vdash dup\ nil\ nil}
    \tag{\sc Nil-Dup}
    \]
    \[
    \frac{\vdash dup\ l_1\ l_2}
    {\vdash dup\ cons(n,l_1)\ cons(n,cons(n,l_2))}
    \tag{\sc Cons-Dup}
    \]
    Prove: If $\vdash ascend\ l_1$ and $\vdash dup\ l_1\ l_2$ then $\vdash ascend\ l_2$.

\end{enumerate}

\newpage 
\mbox{}
\newpage

\end{document}