%----PrefaceImport----%
\documentclass{article}
\usepackage{fancyhdr}
\usepackage[a4paper,margin=1in,headsep=25pt]{geometry}
\usepackage{lipsum,hyperref}
\usepackage{enumerate,fullpage,proof}
\usepackage[fontsize=12pt]{fontsize}
\usepackage{amsmath,amscd,amsbsy,amssymb,latexsym,url,bm,amsthm}
\usepackage{epsfig,graphicx,subfigure}
\usepackage{listings}
\usepackage[usenames]{xcolor}

\newtheorem{thm}{Theorem}
\newtheorem{lemma}[thm]{Lemma}

\pagestyle{fancy}
\pagenumbering{Alph}
\setlength{\headheight}{36.0pt}
\headsep = 25pt
\fancyhf{}
\lhead{CSE 3302/5307 Programming Language Concepts}
\rhead{Homework10:Type Inference II}
\lfoot{2023 Kenny Zhu}
%----Documentation----%
\begin{document}

\title{CSE 3302/5307 Programming Language Concepts}
\author{Homework10 - Fall 2023}
\date{Due Date: Nev.4, 2023, 11:59p.m. Central Time}
\maketitle
\thispagestyle{fancy}

%----Homeworks----%

\section*{Problem1 - 30\%}

Prove the Lemma: If $(S,q) \rightarrow (S', q')$ then:
\begin{itemize}
    \item T is complete for $(S, q)$ iff T is complete for $(S',q')$
    
    \item T is principal for $(S,q)$ iff T is principal for $(S',q')$
\end{itemize}


\section*{Problem2 - 40\%}

\begin{enumerate}[(a)]
    
    \item Give the detailed derivation of the following expressions and obtain the set of equations, then solve these equations by unification algorithm to get the principle solution and give the universal polymorphic types:
        \begin{verbatim}
		let x = inr (5::4::3) in 
		case x of inl y => y.1 + y.2 | 
		          inr y => (case y of nil => 0 | h::l => h)
		\end{verbatim}
\end{enumerate}

\section*{Problem3 - 30\%}
Show why type checking let expression using [t-LetPoly] is exponential in time and give an amortised linear implementation of let polymorphism instead.	

\end{document}