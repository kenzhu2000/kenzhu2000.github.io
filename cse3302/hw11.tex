%----PrefaceImport----%
\documentclass{article}
\usepackage{fancyhdr}
\usepackage[a4paper,margin=1in,headsep=25pt]{geometry}
\usepackage{lipsum,hyperref}
\usepackage{enumerate,fullpage,proof}
\usepackage[fontsize=12pt]{fontsize}
\usepackage{amsmath,amscd,amsbsy,amssymb,latexsym,url,bm,amsthm}
\usepackage{epsfig,graphicx,subfigure}
\usepackage[usenames]{xcolor}
\usepackage{tcolorbox}
\usepackage{amsmath}
\usepackage{listings}
\usepackage{xcolor}
\usepackage{enumitem}   % For customizing the enumerate environment

\newtheorem{thm}{Theorem}
\newtheorem{lemma}[thm]{Lemma}

\pagestyle{fancy}
\pagenumbering{Alph}
\setlength{\headheight}{36.0pt}
\headsep = 25pt
\fancyhf{}
\lhead{CSE 3302/5307 Programming Language Concepts}
\rhead{Homework 11: Type Inference II}
\lfoot{2025 Kenny Zhu Wonjun Park}
\rfoot{Plagiarism will not be tolerated. We are here to help.}

\lstset{
  language=[Objective]Caml,
  basicstyle=\ttfamily,
  keywordstyle=\color{blue},
  commentstyle=\color{gray},
  stringstyle=\color{green},
  showstringspaces=false,
  numbers=left,
  numberstyle=\tiny,
  breaklines=true,
  frame=single
}

% New command for blank spaces after questions
\newcommand{\answerbox}{
    \vspace{7cm} % Adjust the space size as needed
}
\newcommand{\answerboxbig}{
    \vspace{20cm} % Adjust the space size as needed
}
\newcommand{\answerboxsmall}{
    \vspace{3cm} % Adjust the space size as needed
}

\newcommand{\studentinfo}{
    $$\begin{array}{cc}
        \noindent \text{Name:} \underline{\hspace{5cm}} &
            \text{UTA ID:} \underline{\hspace{5cm}}\\
    \end{array}$$
}

\usepackage{listings}
\usepackage{xcolor}
\usepackage{placeins}

\lstset{
    language=C,
    basicstyle=\ttfamily\footnotesize,
    keywordstyle=\color{blue},
    commentstyle=\color{gray},
    stringstyle=\color{orange},
    numbers=none, % Hides line numbers
    showstringspaces=false,
    breaklines=true,
    frame=single,
    rulecolor=\color{black},
}


%----Documentation----%
\begin{document}

\title{CSE 3302/5307 Programming Language Concepts}
\author{Homework 11 - Fall 2025}
\date{Due Date: Nov. 3, 2025, 9:00PM Central Time}
\maketitle
\thispagestyle{fancy}

\studentinfo

%----Homeworks----%

\section*{Problem1 - 20\%}

Specify the inference rules that could be applied by the unification algorithm in a transition $(S,q)\rightarrow(S',q)$ and number each.

\answerboxbig

\section*{Problem2 - 30\%}

Use the unification algorithm to solve each of the following sets of constraints. In each step of the solution, mention the inference rule used.

\begin{enumerate}
    \item $\{X = Int,\ Y = X\rightarrow X\, Z=Z\}$
    \item $\{Int\rightarrow Int = Z\rightarrow X\}$
    \item $\{Z\rightarrow Y = Y\rightarrow X,\ X = U\rightarrow W\}$
    \item $\{Int = Int\rightarrow X\}$
    \item $\{\}$
\end{enumerate} 

\answerboxbig

\section*{Problem3 - 50\%}

Prove the Lemma: If $(S,q) \rightarrow (S', q')$ then:
\begin{itemize}
    \item T is complete for $(S, q)$ iff T is complete for $(S',q')$
    
    \item T is principal for $(S,q)$ iff T is principal for $(S',q')$
\end{itemize}

You can use the following lemma without proof:

\begin{lemma}\label{lemma1}
    If $T(m) = T(n), T|=q$, then $T|=q[n/m]$
\end{lemma}

\begin{lemma}\label{lemma2}
    If $T(a) = T(s), T <= S$, then $T <= [a=s]\circ S$
\end{lemma}

\answerboxbig


\end{document}